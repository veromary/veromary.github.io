% This file was converted to LaTeX by Writer2LaTeX ver. 1.4
% see http://writer2latex.sourceforge.net for more info
\documentclass[a5paper]{article}
%\usepackage[ascii]{inputenc}
%\usepackage[T1]{fontenc}
\usepackage{fontspec}
\usepackage{libertine}
\usepackage[english]{babel}
\usepackage{amsmath}
\usepackage{amssymb,amsfonts,textcomp}
\newfontfamily\jarman{JARDOTTY}
\usepackage{color}
\usepackage[top=2cm,bottom=2cm,left=2cm,right=2cm,nohead,nofoot]{geometry}
\usepackage{array}
\usepackage{hhline}
\usepackage{hyperref}
\hypersetup{colorlinks=true, linkcolor=blue, citecolor=blue, filecolor=blue, urlcolor=blue}
% Footnote rule
\setlength{\skip\footins}{0.119cm}
\renewcommand\footnoterule{\vspace*{-0.018cm}\setlength\leftskip{0pt}\setlength\rightskip{0pt plus 1fil}\noindent\textcolor{black}{\rule{0.25\columnwidth}{0.018cm}}\vspace*{0.101cm}}

\newcommand\textjarman[1]{{\jarman #1}}
\newcommand\answer[1]{\textbf{\textit{#1}}}
\newcounter{z}
\setcounter{z}{0}
\newcommand\spaces[1]{ \_\loop \ifnum\value{z} < #1
~\_%
\stepcounter{z}%
\repeat%
\setcounter{z}{0}}

\title{}
\begin{document}

\setlength{\parskip}{6pt plus2pt minus2pt}


\noindent Name:

\noindent {\Large Chapter XLVI -- The Eucharist}



1. Christ said “I am the \spaces{7} of life.” 

2. Again: “The \spaces{7}  that  I
will give is My \spaces{7} for the \spaces{7} of the world.” 

3. The  Jews  said:
“How can this man give us his flesh to \spaces{7}?” 

4. Christ replied:  “Amen,
amen I say unto you, except you eat the \spaces{7}  of  the  Son  of  Man  and
drink His \spaces{7}, you shall not have \spaces{7} in you.” 

5.  Again:  For  My
flesh is meat \spaces{7} and My Blood  is  drink  \spaces{7}.”  

6.  Peter  said:
“Lord, to whom shall we go? Thou hast the words of  \spaces{7};  and  we  have
believed and have known that  Thou  art  the  \spaces{7},  the  \spaces{7}.” 

7. 
Christ fulfilled His promise at the \spaces{7}. 

8. He  said  over  the  bread:
“This \spaces{7},” and over the wine: “This \spaces{7}.” 

9. He was  then  making
His last \spaces{7}, and also imposing a \spaces{7}. 

10.  St.  Paul  blamed  his
converts for not discerning in the Eucharist “the  \spaces{7}  of  the  Lord.”


11. The letters in “ikthos” are the initial letters  of  Our  Lord's  title,
\spaces{7}. 

12. This word was used to  comply  with  the  “Discipline  of  the
\spaces{7}. 

13. All heretics who broke away during the  first  thousand  years
(\textjarman{believed}) (\textjarman{rejected}) the doctrine of the Real Presence. 

14.  All  realities
are either substances or \spaces{7}. 

15. The words of consecration change  the
(\textjarman{substance}) (\textjarman{accidents}) of the bread and wine. 

16.  This  change  is  called
\spaces{7}. 

17. The words of  consecration  change  the  (\textjarman{matter  only})  (\textjarman{form
only}) (\textjarman{matter and form}) of the  bread  and  wine.  

18.  God  (\textjarman{can})  \spaces{3}
change a creature into Himself. 

19. God (\textjarman{can}) \spaces{3}  instantly  change  a
monkey into an angel. 

20. After the consecration, God keeps the \spaces{7}  of
the bread and wine in being, and the other accidents exist in  it.  

21.  The
words of  consecration  put  only  \spaces{7}  of  Christ  present  under  the
appearances of the bread; but  the  rest  of  Our  Lord  is  also  there  by
\spaces{7}. 

22. Christ (\textjarman{has}) (\textjarman{has not}) His natural stature in  the  Host. 

23. 
He (\textjarman{is}) \spaces{7} wholly present in every part of the Host. 

24. He  (\textjarman{is})  (\textjarman{is
not}) present circumscriptively in the Eucharist. 

25. There,  He  (\textjarman{has})  (\textjarman{has
not}) physical contact with things about Him. 

26. It (\textjarman{is})  (\textjarman{is  not})  correct
to say that He comes down from heaven to the altar. 

27. In the Eucharist  He
(\textjarman{is}) \spaces{7} distant from Himself in heaven. 

28. The special effect of  the
Eucharist is to increase one of  the  supernatural  virtues.  Which  is  it?
\spaces{7}. 

29. With  it,  it  increases  the  Gifts  which  give  us  infused
\spaces{7}. 

30. The Eucharist is an infinitely precious treasure,  because  it
is \spaces{7} under the appearances of bread and wine.

\newpage


1. Christ said ``I am the \answer{Bread} of life.'' 2. Again: ``The \answer{Bread}  that  I
will give is My \answer{Flesh} for the \answer{life} of the world.'' 3. The  Jews  said:
``How can this man give us his flesh to \answer{eat}?'' 4. Christ replied:  ``Amen,
amen I say unto you, except you eat the \answer{flesh}  of  the  Son  of  Man  and
drink His \answer{blood}, you shall not have \answer{life} in you.'' 5.  Again:  ``For  My
flesh is meat \answer{indeed} and My Blood  is  drink  \answer{indeed}.''  6.  Peter  said:
“Lord, to whom shall we go? Thou hast the words of  \answer{eternal life};  and  we  have
believed and have known that  Thou  art  the  \answer{Christ},  the  \answer{Son of God}.”  7.
Christ fulfilled His promise at the \answer{Last Supper}. 8. He  said  over  the  bread:
“This \answer{is My Body},” and over the wine: “This \answer{is My Blood}.” 9. He was  then  making
His last \answer{will}, and also imposing a \answer{law}. 10.  St.  Paul  blamed  his
converts for not discerning in the Eucharist “the  \answer{Body}  of  the  Lord.”
11. The letters in “ikthos” are the initial letters  of  Our  Lord's  title,
\answer{Christ, Son of God, Saviour}. 12. This word was used to  comply  with  the  ``Discipline  of  the
\answer{Secret}.'' 13. All heretics who broke away during the  first  thousand  years
(\answer{believed}) (rejected) the doctrine of the Real Presence. 14.  All  realities
are either substances or \answer{accidents}. 15. The words of consecration change  the
(\answer{substance}) (accidents) of the bread and wine. 16.  This  change  is  called
\answer{transubstantiation}. 17. The words of  consecration  change  the  (matter  only)  (form
only) (\answer{matter and form}) of the  bread  and  wine.  18.  God  \answer{cannot}
change a creature into Himself. 19. God \answer{can}  instantly  change  a
monkey into an angel. 20. After the consecration, God keeps the \answer{quantity}  of
the bread and wine in being, and the other accidents exist in  it.  21.  The
words of  consecration  put  only  \answer{Body}  of  Christ  present  under  the
appearances of the bread; but  the  rest  of  Our  Lord  is  also  there  by
\answer{concomitance}. 22. Christ (has) (\answer{has not}) His natural stature in  the  Host.  23.
He \answer{is} wholly present in every part of the Host. 
24. He  (is)  (\answer{is not}) present circumscriptively in the Eucharist. 
25. There,  He  (has)  (\answer{has not}) physical contact with things about Him. 
26. It (is) (\answer{is not})  correct
to say that He comes down from heaven to the altar. 27. In the Eucharist  He
(is) \answer{is not} distant from Himself in heaven. 28. The special effect of  the
Eucharist is to increase one of  the  supernatural  virtues.  Which  is  it?
\answer{Charity}. 29. With  it,  it  increases  the  Gifts  which  give  us  infused
\answer{contemplation}. 30. The Eucharist is an infinitely precious treasure,  because  it
is \answer{God} under the appearances of bread and wine.



\end{document}
