% This file was converted to LaTeX by Writer2LaTeX ver. 1.4
% see http://writer2latex.sourceforge.net for more info
\documentclass[a4paper,14pt]{memoir}
%\usepackage[ascii]{inputenc}
%\usepackage[T1]{fontenc}
\usepackage{fontspec}
\usepackage{libertine}
\usepackage[english]{babel}
\usepackage{amsmath}
\usepackage{amssymb,amsfonts,textcomp}
\newfontfamily\jarman{JARDOTTY}
\usepackage{color}
\usepackage[top=2cm,bottom=2cm,left=2cm,right=2cm,nohead,nofoot]{geometry}
\usepackage{array}
\usepackage{hhline}
\usepackage{hyperref}
\hypersetup{colorlinks=true, linkcolor=blue, citecolor=blue, filecolor=blue, urlcolor=blue}
% Footnote rule
\setlength{\skip\footins}{0.119cm}
\renewcommand\footnoterule{\vspace*{-0.018cm}\setlength\leftskip{0pt}\setlength\rightskip{0pt plus 1fil}\noindent\textcolor{black}{\rule{0.25\columnwidth}{0.018cm}}\vspace*{0.101cm}}

\newcommand\textjarman[1]{{\jarman #1}}
\newcommand\answer[1]{\textbf{\textit{#1}}}
\newcounter{z}
\setcounter{z}{0}
\newcommand\spaces[1]{ \_\loop \ifnum\value{z} < #1
~\_%
\stepcounter{z}%
\repeat%
\setcounter{z}{0}}

\title{}
\begin{document}

\setlength{\parskip}{6pt plus2pt minus2pt}


\noindent Name:

\noindent {\Large Chapter L -- Orders}


1. The two social sacraments are those of \spaces{7} and  \spaces{7}.  

2.  Christ
instituted the priesthood at the \spaces{7}. 

3. The priestly  character  finds
its complement in the \spaces{7}. 

4. The Apostles not  only  ordained  priests
but also consecrated many of them \spaces{7}. 

5. Scripture speaks  of  \spaces{7}
as well as of priests and bishops. 

6. The four minor  orders  are 
\begin{itemize}
\item \spaces{7},
\item \spaces{7},
\item \spaces{7}, 
\item \spaces{7}.  
\end{itemize}

7.  The  three  major  orders  are 
\begin{itemize}
\item \spaces{7},
\item \spaces{7}, 
\item \spaces{7}.  
\end{itemize}


8. A sub-deacon has to remain a \spaces{7}  for  life  and
say the \spaces{7} daily. 

9. A priest represents Christ inasmuch as He is  our
\spaces{7}; a bishop, inasmuch as He is  \spaces{7}.  

10.  It  is  the duty of a  (\textjarman{priest})
(\textjarman{bishop}) to rule a diocese. 

11. The matter in  the  major  orders  is
\spaces{7}. 

12. A priest is a \spaces{7}  between  God  and  men.  

13.  Christ's
priesthood is the most excellent possible, on  account  of  His  union  with
\spaces{7}, with \spaces{7} and with \spaces{7}. 

14. The Catholic priesthood is  a
sharing in that of \spaces{7}. 

15. Its dignity is also seen  in  the  \spaces{7}
at Mass, in the \spaces{7} in the Confessional, and  in  the  \spaces{7}  graces
given by ordination. 

\eject

16. The three conditions required for  validity  in  an
ordination to the priesthood are 
\begin{enumerate}
\item \spaces{7} 
\item \spaces{7}  
\item \spaces{7}.  
\end{enumerate}


17.  Five
conditions required for licity are 
\begin{enumerate}
\item \spaces{7} 
\item \spaces{7} 
\item \spaces{7} 
\item \spaces{7}  
\item \spaces{7}.  
\end{enumerate}


18. A priestly vocation is an invitation to receive  \spaces{7}.  It
comes from \spaces{7} through a \spaces{7}. 

19. Apart from a  bishop's  call  to
receive ordination, (\textjarman{any}) (\textjarman{no}) seminary student has a right to be  ordained.


20. Christ says to His priests: ``You have not chosen Me,  but  I  \spaces{7}.''


21. St. Paul says: ``Nor doth anyone take the honour to himself, but he  that
is \spaces{7} as Aaron was.''

22. To go to a seminary, a  student  should  have
these four  qualifications:  
\begin{enumerate}
\item \spaces{7} 
\item \spaces{7} 
\item \spaces{7}  
\item \spaces{7}.  
\end{enumerate}

23.  A
religious vocation is an invitation to take the three \spaces{7}. It is  given
by \spaces{7} through a \spaces{7}. 

24. The vows remove the chief  obstacles  to
\spaces{7}. 

25. Vocations are best fostered by a truly Christian \spaces{7}.

\newpage

1. The two social sacraments are those of \answer{} and  \answer{}.  
2.  Christ
instituted the priesthood at the \answer{}. 
3. The priestly  character  finds
its complement in the \answer{}. 
4. The Apostles not  only  ordained  priests
but also consecrated many of them \answer{}. 
5. Scripture speak  of  \answer{}
as well as of priests and bishops. 
6. The four minor  orders  are  \answer{},
\answer{}, \answer{}, \answer{}.  
7.  The  three  major  orders  are  \answer{},
\answer{}, \answer{}. 
8. A sub-deacon has to remain a \answer{}  for  life  and
say the \answer{} daily. 
9. A priest represents Christ inasmuch as He is  our
\answer{}; a bishop, inasmuch as He is  \answer{}.  
10.  It  is  a  (priest's)
(bishop's) duty to rule a diocese. 
11. The matter in  the  major  orders  is
\answer{}. 
12. A priest is a \answer{}  between  God  and  men.  
13.  Christ's
priesthood is the most excellent possible, on  account  of  His  union  with
\answer{}, with \answer{} and with \answer{}. 
14. The Catholic priesthood is  a
sharing in that of \answer{}. 
15. Its dignity is also seen  in  the  \answer{}
at Mass, in the \answer{} in the Confessional, and  in  the  \answer{}  graces
given by ordination. 
16. The three conditions required for  validity  in  an
ordination to the priesthood are  \answer{},  \answer{},  \answer{}.  
17.  Five
conditions required for licity are \answer{}, \answer{},  \answer{},  \answer{},
\answer{}. 
18. A priestly vocation is an invitation to receive  \answer{}.  It
comes from \answer{} through a \answer{}. 
19. Apart from a  bishop's  call  to
receive ordination, (any) (no) seminary student has a right to be  ordained.

20. Christ says to His priests: ``You have not chosen Me,  but  I  \answer{}.''

21. St. Paul says: ``Nor doth anyone take the honour to himself, but he  that
is \answer{} as Aaron was.''
22. To go to a seminary, a  student  should  have
these four  qualifications:  \answer{}  \answer{}  \answer{}  \answer{}.  
23.  A
religious vocation is an invitation to take the three \answer{}. It is  given
by \answer{} through a \answer{}. 
24. The vows remove the chief  obstacles  to
\answer{}. 25. Vocations are best fostered by a truly Christian \answer{}.


\end{document}
