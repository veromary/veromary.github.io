% This file was converted to LaTeX by Writer2LaTeX ver. 1.4
% see http://writer2latex.sourceforge.net for more info
\documentclass[a4paper]{article}
\usepackage[ascii]{inputenc}
\usepackage[T1]{fontenc}
\usepackage[english]{babel}
\usepackage{amsmath}
\usepackage{amssymb,amsfonts,textcomp}
\usepackage{color}
\usepackage[top=2cm,bottom=2cm,left=2cm,right=2cm,nohead,nofoot]{geometry}
\usepackage{array}
\usepackage{hhline}
\usepackage{hyperref}
\hypersetup{colorlinks=true, linkcolor=blue, citecolor=blue, filecolor=blue, urlcolor=blue}
% Footnote rule
\setlength{\skip\footins}{0.119cm}
\renewcommand\footnoterule{\vspace*{-0.018cm}\setlength\leftskip{0pt}\setlength\rightskip{0pt plus 1fil}\noindent\textcolor{black}{\rule{0.25\columnwidth}{0.018cm}}\vspace*{0.101cm}}
\title{}
\begin{document}
Name:

Chapter XLII -- The Sacraments

1. A sacrament is an \ \_ \_ \_ \_ \_ \_ \_ \_ \_ \ sign of \ \_ \_ \_ \_ \_ \_ \_ , instituted by \ \_ \_ \_ \_ \_ \_ \_ .

2. The Church (can) (cannot) make a new sacrament.

3. In the sacraments, grace which is invisible is \ \_ \_ \_ \_ \_ \_ \_ \ connected with material things which we can see and hear.

4. A sacrament is made up of matter and \ \_ \_ \_ \_ \_ \_ \_ , united to make one efficacious \ \_ \_ \_ \_ \_ \_ \_ \ of grace.

5. We (are) (are not) allowed to alter these substantially.

6. God is the \ \_ \_ \_ \_ \_ \_ \_ \_ \_ Cause of the grace produced by the sacraments; they are its \ \_ \_ \_ \_ \_ \_ \_ \_ \_ \ causes.

7. They cause grace by (their use) (dispositions they arouse in us).

8. The external ceremony is called the (sign only) (thing and sign) (thing only).

9. If valid, it must produce in the soul a supernatural reality called the (sign only) (thing and sign) (thing only).

10. This then produces the (sign only) (thing and sign) (thing only), which is \ \_ \_ \_ \_ \_ \_ \_ \_ \ grace, provided there is no \_ \_ \_ \_ \_ \_ \_ .

11. In baptism, confirmation and holy orders, the thing and sign corresponds to the \_ \_ \_ \_ \_ \_ \_ \_ \ given by these sacraments.

12. One who receives baptism validly but unfruitfully (can) (cannot) get grace from it later in life.

13. If there is no sacrament at all, it is said to be (invalid) (valid, but not fruitful) (valid and fruitful).

14. A man gets married validly, but in mortal sin. Next day he makes an act of perfect contrition. Does this marriage then give him grace? (Yes) (No).

15. In the Eucharist, the Body and Blood of Christ are the (sign only) (thing and sign) (thing only) of the sacrament.

16. In marriage, the thing and sign is the \ \_ \_ \_ \_ \_ \_ \_ \ \ \_ \_ \_ \_ .

17. In penance, it is \ \_ \_ \_ \_ \_ \_ \ \ \_ \_ \_ \ \ \_ \_ \_ \ \ which is a sharing in that experienced by Christ in Gethsemane.

18. The sacramentals are instituted by (Christ) (the Church).

19. They cause grace as efficient causes. (True) (False).

20. Their efficacy depends on the power of the \ \_ \_ \_ \_ \_ \_ \_ \ of the Church. 


\bigskip

\clearpage
1. A sacrament is an (\textbf{\textit{efficacious}}) sign of (\textbf{\textit{grace}}), instituted by (\textbf{\textit{Christ}}). 2. The Church (can) (\textbf{\textit{cannot}}) make a new sacrament. 3. In the sacraments, grace which is invisible is (\textbf{\textit{infallibly}}) connected with material things which we can see and hear. 4. A sacrament is made up of matter and (\textbf{\textit{form}}), united to make one efficacious (\textbf{\textit{sign}}) of grace. 5. We (are) (\textbf{\textit{are not}}) allowed to alter these substantially. 6. God is the (\textbf{\textit{Principal}}) Cause of the grace produced by the sacraments; they are its (\textbf{\textit{instrumental}}) causes. 7. They cause grace by (\textbf{\textit{their use}}) (dispositions they arouse in us). 8. The external ceremony is called the (\textbf{\textit{sign only}}) (thing and sign) (thing only). 9. If valid, it must produce in the soul a supernatural reality called the (sign only) (\textbf{\textit{thing and sign}}) (thing only). 10. This then produces the (sign only) (thing and sign) (\textbf{\textit{thing only}}), which is (\textbf{\textit{sacramental}}) grace, provided there is no (\textbf{\textit{obstacle}}). 11. In baptism, confirmation and holy orders, the thing and sign corresponds to the (\textbf{\textit{character}}) given by these sacraments. 12. One who receives baptism validly but unfruitfully (\textbf{\textit{can}}) (cannot) get grace from it later in life. 13. If there is no sacrament at all, it is said to be (\textbf{\textit{invalid}}) (valid, but not fruitful) (valid and fruitful). 14. A man gets married validly, but in mortal sin. Next day he makes an act of perfect contrition. Does this marriage then give him grace? (\textbf{\textit{Yes}}) (No). 15. In the Eucharist, the Body and Blood of Christ are the (sign only) (\textbf{\textit{thing and sign}}) (thing only) of the sacrament. 16. In marriage, the thing and sign is the (\textbf{\textit{marriage bond}}). 17. In penance, it is (\textbf{\textit{sorrow for sin}}) which is a sharing in that experienced by Christ in Gethsemane. 18. The sacramentals are instituted by (Christ) (\textbf{\textit{the Church}}). 19. They cause grace as efficient causes. (True) (\textbf{\textit{False}}). 20. Their efficacy depends on the power of the (\textbf{\textit{prayer}}) of the Church. 
\end{document}
