% This file was converted to LaTeX by Writer2LaTeX ver. 1.4
% see http://writer2latex.sourceforge.net for more info
\documentclass[a4paper,14pt]{memoir}
%\usepackage[ascii]{inputenc}
%\usepackage[T1]{fontenc}
\usepackage{fontspec}
\usepackage{libertine}
\usepackage[english]{babel}
\usepackage{amsmath}
\usepackage{amssymb,amsfonts,textcomp}
\newfontfamily\jarman{JARDOTTY}
\usepackage{color}
\usepackage[top=2cm,bottom=2cm,left=2cm,right=2cm,nohead,nofoot]{geometry}
\usepackage{array}
\usepackage{hhline}
\usepackage{hyperref}
\hypersetup{colorlinks=true, linkcolor=blue, citecolor=blue, filecolor=blue, urlcolor=blue}
% Footnote rule
\setlength{\skip\footins}{0.119cm}
\renewcommand\footnoterule{\vspace*{-0.018cm}\setlength\leftskip{0pt}\setlength\rightskip{0pt plus 1fil}\noindent\textcolor{black}{\rule{0.25\columnwidth}{0.018cm}}\vspace*{0.101cm}}

\newcommand\textjarman[1]{{\jarman #1}}
\newcommand\answer[1]{\textbf{\textit{#1}}}
\newcounter{z}
\setcounter{z}{0}
\newcommand\spaces[1]{\rule{0pt}{18pt} \_\loop \ifnum\value{z} < #1
~\_%
\stepcounter{z}%
\repeat%
\setcounter{z}{0}}

\title{}
\begin{document}

\setlength{\parskip}{6pt plus2pt minus2pt}


\noindent Name:

\noindent {\Large Chapter LII -- Death and Judgement}




1. Scripture says: ``The \spaces{7} shall return to its earth whence it was;  and
the \spaces{7} return to God who gave it.''

2. This shows that man  is  a  \spaces{7}
being, made up  of  a  material  \spaces{7}  and  a  spiritual  \spaces{7}.  

3.  The
dissolution of this composite is called \spaces{7}. 

4. Our Lord  says:  ``What  I
say to you, I say to all: `\spaces{7}'.'' 

5. Speaking of one who lived  just  for
this world, Our Lord said: “Thou \spaces{7}! This  night  do  they  require  thy
\spaces{7} of thee; and whose shall these things be which thou  hast  provided?”


6. Death in a state of grace is called \spaces{7} \spaces{9}.  

7.  Death  in  a  state  of
personal mortal sin is called \spaces{7} \spaces{9}. 

8. We (\textjarman{can}) (\textjarman{cannot}) merit to  die  in
a state of grace. 

9. The grace of a happy death can be gained 
\begin{itemize}
\item by \spaces{7},
\item by \spaces{7},
\item and by \spaces{7}.  
\end{itemize}


10. A person who dies without showing  outwardly  any
signs of contrition (\textjarman{does}) (\textjarman{does not}) necessarily lose his soul.  

11.  Death
has three main effects:
\begin{itemize}
\item it \spaces{7},
\item it \spaces{7},
\item and it  \spaces{7}.  
\end{itemize}


12. The  amount
of merit we gain from an act depends almost entirely  on  the  intensity  of
the \spaces{7} with which it is done. 

13. All  who  have  come  to  the  use  of
reason in moral matters will spend eternity either in \spaces{7} or  in  \spaces{7}.


14. At the particular judgement the soul (\textjarman{sees})  (\textjarman{does  not  see})  God.  

15.
This judgement takes place at the moment of \spaces{7}.  

16.  It  implies  three
things, namely 
\begin{itemize}
\item \spaces{7},
\item \spaces{7},
\item and \spaces{7}.  
\end{itemize}


17. This judgement (\textjarman{is})  (\textjarman{is  not})
absolutely final. 

18. The Church provides four means by which we can  purify
our soul completely at death, namely 
\begin{itemize}
\item \spaces{7},
\item \spaces{7},
\item \spaces{7},
\item and \spaces{7}.  
\end{itemize}


19.
St. Paul says: ``For me, to die is \spaces{7}.''

20. I should always act now as  I
shall wish to have acted at the moment of \spaces{7}.


\newpage

1. Scripture says: ``The \answer{dust} shall return to its earth whence it was;  and
the \answer{spirit} return to God who gave it.''
2. This shows that man  is  a  \answer{composite}
being, made up  of  a  material  \answer{body}  and  a  spiritual  \answer{soul}. 
3.  The
dissolution of this composite is called \answer{death}.
4. Our Lord  says:  ``What  I
say to you, I say to all: \answer{Watch!}.''
5. Speaking of one who lived  just  for
this world, Our Lord said: ``Thou \answer{fool}! This  night  do  they  require  thy
\answer{soul} of thee; and whose shall these things be which thou  hast  provided?''
6. Death in a state of grace is called  \answer{final perseverence}. 
7.  Death  in  a  state  of
personal mortal sin is called \answer{final impenitence}.
8. We (can) (\answer{cannot}) merit to  die  in
a state of grace.
9. The grace of a happy death can be gained by \answer{fervent prayer},  by
\answer{Our Lady's intercession}, and by \answer{having Masses said}.
10. A person who dies without showing  outwardly  any
signs of contrition (does) (\answer{does not}) necessarily lose his soul. 
11.  Death
has three main effects: (a).    \answer{It strips from us all worldly goods forever.}
 (b).    \answer{It ends forever our chance of merit.}
 (c).    \answer{It brings before us the dread alternative: heaven or hell forever.}
12. The  amount
of merit we gain from an act depends almost entirely  on  the  intensity  of
the \answer{charity} with which it is done.
13. All  who  have  come  to  the  use  of
reason in moral matters will spend eternity either in \answer{heaven} or  in  \answer{hell}.
14. At the particular judgement the soul (sees)  (\answer{does  not  see})  God.
15.
This judgement takes place at the moment of \answer{death}. 
16.  It  implies  three
things, namely \answer{examination, sentence, execution.
}
17. This judgement (\answer{is})  (is  not)
absolutely final.
18. The Church provides four means by which we can  purify
our soul completely at death, namely \answer{Confession}, \answer{Communion}, \answer{Last Anointing} and \answer{Last Blessing}.  
19.
St. Paul says: ``For me, to die is \answer{gain}.''
20. I should always act now as  I
shall wish to have acted at the moment of \answer{death}.

\end{document}

