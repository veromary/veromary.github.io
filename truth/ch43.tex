% This file was converted to LaTeX by Writer2LaTeX ver. 1.4
% see http://writer2latex.sourceforge.net for more info
\documentclass[a5paper]{article}
\usepackage[ascii]{inputenc}
\usepackage[T1]{fontenc}
\usepackage{fontspec}
\usepackage{libertine}
\usepackage[english]{babel}
\usepackage{amsmath}
\usepackage{amssymb,amsfonts,textcomp}
\newfontfamily\jarman{JARDOTTY}
\usepackage{color}
\usepackage[top=2cm,bottom=2cm,left=2cm,right=2cm,nohead,nofoot]{geometry}
\usepackage{array}
\usepackage{hhline}
\usepackage{hyperref}
\hypersetup{colorlinks=true, linkcolor=blue, citecolor=blue, filecolor=blue, urlcolor=blue}
% Footnote rule
\setlength{\skip\footins}{0.119cm}
\renewcommand\footnoterule{\vspace*{-0.018cm}\setlength\leftskip{0pt}\setlength\rightskip{0pt plus 1fil}\noindent\textcolor{black}{\rule{0.25\columnwidth}{0.018cm}}\vspace*{0.101cm}}

\newcommand\textjarman[1]{{\jarman #1}}
\newcommand\answer[1]{\textbf{\textit{#1}}}

\title{}
\begin{document}

\setlength{\parskip}{6pt plus2pt minus2pt}


\noindent Name:

\noindent {\Large Chapter XLIII -- Baptism}


1. Baptism is a new supernatural  \_~\_~\_~\_~\_~\_~\_~; penance is a  \_~\_~\_~\_~\_~\_~\_  from the dead.

2. St. Paul says that after baptism our body is not for uncleanness, but “for the  \_~\_~\_~\_~\_~\_~\_~.” It is in fact consecrated to  \_~\_~\_~\_~\_~\_~\_~.

3. Christ said to Nicodemus: “Unless a man be  \_~\_~\_~\_~\_~\_~\_  again of  \_~\_~\_~\_~\_~\_~\_  and the  \_~\_~\_~\_~\_~\_~\_~, he cannot enter the  \_~\_~\_~\_~\_~\_~\_~.”

4. He said to His Apostles: “Going, therefore, teach all nations,  \_~\_~\_~\_~\_~\_~\_  them in the name of the Father and of the Son and of the Holy Ghost.”

5. It was the Council of  \_~\_~\_~\_~\_~\_~\_  that defined baptism as one of the sacraments.

6. The remote matter of baptism is  \_~\_~\_~\_~\_~\_~\_  and  \_~\_~\_~\_~\_~\_~\_  water.

7. The proximate matter is the  \_~\_~\_~\_~\_~\_~\_  of the  \_~\_~\_~\_~\_~\_~\_  on a person to be baptised.

8. This washing may be done by  \_~\_~\_~\_~\_~\_~\_~, or by  \_~\_~\_~\_~\_~\_~\_~, or by  \_~\_~\_~\_~\_~\_~\_~.

9. The form in baptism is:  \_~\_~\_~\_~\_~\_~\_~.

10. If one man poured the water and another said the words, this baptism (\textjarman{would}) (\textjarman{would not}) be invalid.

11. One (\textjarman{can}) (\textjarman{cannot}) baptise oneself.

12. Baptism cannot be repeated because it is a supernatural  \_~\_~\_~\_~\_~\_~\_~; moreover, Christ  \_~\_~\_~\_~\_~\_~\_  and  \_~\_~\_~\_~\_~\_~\_  only once; finally, it imprints a  \_~\_~\_~\_~\_~\_~\_  which is  \_~\_~\_~\_~\_~\_~\_~.

13. There are three kinds of baptism, namely  \_~\_~\_~\_~\_~\_~\_~,  \_~\_~\_~\_~\_~\_~\_  and  \_~\_~\_~\_~\_~\_~\_~.

14. Of these  \_~\_~\_~\_~\_~\_~\_  alone is a sacrament.

15. Which form of baptism carries with it the grace of final perseverance?  \_~\_~\_~\_~\_~\_~\_~.

16. Baptism may also be solemn or it may be  \_~\_~\_~\_~\_~\_~\_~.

17. The ordinary minister for solemn baptism is  \_~\_~\_~\_~\_~\_~\_~; the extraordinary, is  \_~\_~\_~\_~\_~\_~\_~.

18. Can a doctor who is an atheist administer baptism validly? (\textjarman{Yes}) (\textjarman{No}).

19. How many god-parents should be present at a baptism where possible? Not more than  \_~\_~\_~\_~\_~\_~\_~; and at least  \_~\_~\_~\_~\_~\_~\_~.

20. Baptism sets up between the baptised and both the minister and the sponsor a spiritual relationship which is an impediment to  \_~\_~\_~\_~\_~\_~\_~.

21. Christ says: “Suffer little  \_~\_~\_~\_~\_~\_~\_  to come unto Me, and forbid them not.” From this, does it seem that He was opposed to infant baptism? (\textjarman{Yes}) (\textjarman{No}).

22. Complete education is impossible without grace, for man is destined for  \_~\_~\_~\_~\_~\_~\_~.

23. Children have a right to be protected by their parents from all evils,  \_~\_~\_~\_~\_~\_~\_  as well as physical.

24. When received validly and fruitfully, baptism by water removes (\textjarman{all}) (\textjarman{some}) guilt of (\textjarman{all}) (\textjarman{some}) sins.

25. It also removes (\textjarman{all}) (\textjarman{some}) of the temporal punishment due to sin.

26. It also gives  \_~\_~\_~\_~\_~\_~\_  grace to the soul; and with it all the  \_~\_~\_~\_~\_~\_~\_  virtues and all the  \_~\_~\_~\_~\_~\_~\_  of the Holy Ghost.

27. Its special sacramental graces are those of union,  \_~\_~\_~\_~\_~\_~\_~, and  \_~\_~\_~\_~\_~\_~\_~.

28. The character it gives incorporates into the  \_~\_~\_~\_~\_~\_~\_  Body of Christ.

29. It also enables layfolk to be ministers of the sacrament of  \_~\_~\_~\_~\_~\_~\_~.

30. It gives us, too, a share in the virtues displayed by Christ as a  \_~\_~\_~\_~\_~\_~\_~, namely, in His  \_~\_~\_~\_~\_~\_~\_~, His  \_~\_~\_~\_~\_~\_~\_  and His  \_~\_~\_~\_~\_~\_~\_~.

31. By it, again, we are made subjects of Christ, the  \_~\_~\_~\_~\_~\_~\_~; hence we are obliged to be utterly loyal to Him and to His Church. 



\newpage

1. Baptism is a new supernatural \answer{birth}; penance is a \answer{resurrection} from the dead.
2. St. Paul says that after baptism our body is not for uncleanness, but “for the \answer{Lord}.” It is in fact consecrated to \answer{Christ}.
3. Christ said to Nicodemus: “Unless a man be \answer{born} again of \answer{water} and the \answer{spirit}, he cannot enter the \answer{Kingdom of God}.”
4. He said to His Apostles: “Going, therefore, teach all nations, \answer{baptising} them in the name of the Father and of the Son and of the Holy Ghost.”
5. It was the Council of \answer{Trent} that defined baptism as one of the sacraments.
6. The remote matter of baptism is \answer{true} and \answer{natural} water.
7. The proximate matter is the \answer{washing} of the \answer{skin} on a person to be baptised.
8. This washing may be done by \answer{immersion}, or by \answer{sprinkling}, or by \answer{pouring}.
9. The form in baptism is: \answer{I baptise thee, in the name of the Father and of the Son and of the Holy Ghost}.
10. If one man poured the water and another said the words, this baptism \answer{would} be invalid.
11. One \answer{cannot} baptise oneself.
12. Baptism cannot be repeated because it is a supernatural \answer{birth}; moreover, Christ \answer{died} and \answer{rose} only once; finally, it imprints a \answer{character} which is \answer{indelible}.
13. There are three kinds of baptism, namely \answer{water}, \answer{desire} and \answer{blood}.
14. Of these \answer{water} alone is a sacrament.
15. Which form of baptism carries with it the grace of final perseverance? \answer{Blood}.
16. Baptism may also be solemn or it may be \answer{private}.
17. The ordinary minister for solemn baptism is \answer{a priest}; the extraordinary, is \answer{a deacon}.
18. Can a doctor who is an atheist administer baptism validly? \answer{Yes}.
19. How many god-parents should be present at a baptism where possible? Not more than \answer{two}; and at least \answer{one}.
20. Baptism sets up between the baptised and both the minister and the sponsor a spiritual relationship which is an impediment to \answer{marriage}.
21. Christ says: “Suffer little \answer{children} to come unto Me, and forbid them not.” From this, does it seem that He was opposed to infant baptism? \answer{No}.
22. Complete education is impossible without grace, for man is destined for \answer{heaven}.
23. Children have a right to be protected by their parents from all evils, \answer{moral} as well as physical.
24. When received validly and fruitfully, baptism by water removes (\answer{all}) (some) guilt of (\answer{all}) (some) sins.
25. It also removes (\answer{all}) (some) of the temporal punishment due to sin.
26. It also gives \answer{sanctifying} grace to the soul; and with it all the \answer{supernatural} virtues and all the \answer{Gifts} of the Holy Ghost.
27. Its special sacramental graces are those of union, \answer{}, and \answer{}.
28. The character it gives incorporates into the \answer{Mystical} Body of Christ.
29. It also enables layfolk to be ministers of the sacrament of \answer{Marriage}.
30. It gives us, too, a share in the virtues displayed by Christ as a \answer{child}, namely, in His \answer{obedience}, His \answer{humility} and His \answer{filial confidence in His Heavenly Father}.
31. By it, again, we are made subjects of Christ, the \answer{King of kings}; hence we are obliged to be utterly loyal to Him and to His Church. 

\end{document}
