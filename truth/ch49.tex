% This file was converted to LaTeX by Writer2LaTeX ver. 1.4
% see http://writer2latex.sourceforge.net for more info
\documentclass[a4paper,14pt]{memoir}
%\usepackage[ascii]{inputenc}
%\usepackage[T1]{fontenc}
\usepackage{fontspec}
\usepackage{libertine}
\usepackage[english]{babel}
\usepackage{amsmath}
\usepackage{amssymb,amsfonts,textcomp}
\newfontfamily\jarman{JARDOTTY}
\usepackage{color}
\usepackage[top=2cm,bottom=2cm,left=2cm,right=2cm,nohead,nofoot]{geometry}
\usepackage{array}
\usepackage{hhline}
\usepackage{hyperref}
\hypersetup{colorlinks=true, linkcolor=blue, citecolor=blue, filecolor=blue, urlcolor=blue}
% Footnote rule
\setlength{\skip\footins}{0.119cm}
\renewcommand\footnoterule{\vspace*{-0.018cm}\setlength\leftskip{0pt}\setlength\rightskip{0pt plus 1fil}\noindent\textcolor{black}{\rule{0.25\columnwidth}{0.018cm}}\vspace*{0.101cm}}

\newcommand\textjarman[1]{{\jarman #1}}
\newcommand\answer[1]{\textbf{\textit{#1}}}
\newcounter{z}
\setcounter{z}{0}
\newcommand\spaces[1]{ \_\loop \ifnum\value{z} < #1
~\_%
\stepcounter{z}%
\repeat%
\setcounter{z}{0}}

\title{}
\begin{document}

\setlength{\parskip}{6pt plus2pt minus2pt}


\noindent Name:

\noindent {\Large Chapter XLIX -- Extreme Unction}



1. At death we feel utterly alone, for each of us  is  a  \spaces{7}.  

2.  The
priest can help the dying by the graces given in the sacrament of   \spaces{7} \spaces{7}.


3. St. James says: ``Is any man \spaces{7} among you?  Let  him  bring  in  the
\spaces{7}, and let them \spaces{7} over  him,  \spaces{7}  him  in  the  name  of
\spaces{7}. And the prayer of \spaces{7} shall save the sick man;  and  \spaces{7}
shall raise him up; and, if he be in \spaces{7}, they shall be \spaces{7}  him.''


4. He is here speaking  of  a  sign,  since  he  mentions  an  \spaces{7}  and
\spaces{7}. 

5. It is a sign of grace because it  gives  supernatural  help  to
the sick, and forgives \spaces{7}. 

6. It is an efficacious  sign,  since  this
promise is fulfilled \spaces{7}. 

7. It is instituted by  Christ  since  it  is
administered in His \spaces{7} and with  His  \spaces{7}.  

8.  Hence,  it  is  a
\spaces{7} . 

9. Anglicans can see what the Catholic Church in  England  taught
concerning  Extreme  Unction,  before  the  Reformation,  by   reading   the
Sacerdotal Law of  \spaces{7},  Archbishop  of  \spaces{7},  written  more  than
\spaces{7} years ago. 

10. The remote matter of Extreme  Unction  is  \spaces{7}.


11. Its proximate matter is \spaces{7}. 

12. The parts of  the  body  that  are
anointed are the \spaces{7},  \spaces{7},  \spaces{7},  \spaces{7} and  \spaces{7} . 

13. The long form  is  \spaces{7} \spaces{7} \spaces{7} \spaces{7} \spaces{7}.  

14.  The  short
form is \spaces{7} \spaces{7} \spaces{7} \spaces{7} \spaces{7} .

15. This sacrament  (\textjarman{does})  (\textjarman{does  not})  give  sanctifying
grace. 

16. It forgives sin, provided the  recipient  has  \spaces{7}  for  his
sins. 

17. It removes from the soul the remnants of sin, which are  \spaces{7},
\spaces{7} and \spaces{7} \spaces{7}. 

18. It brings supernatural \spaces{7} and  \spaces{7}  to
the sick. 

19. It (\textjarman{can}) (\textjarman{cannot}) make  the  soul  so  pure  that  it  can  go
straight to heaven. 

20.  It  (\textjarman{does})  (\textjarman{does  not})  regularly  restore  bodily
health by a miracle. 

21. It is foolish to delay too long in  having  Extreme
Unction administered because it  restores  bodily  health  if   \spaces{7} \spaces{7},  if
\spaces{7}  \spaces{7} and if \spaces{7} \spaces{7}. 

22. It is because \spaces{7} \spaces{7}, because  \spaces{7} \spaces{7}  and
because \spaces{7} \spaces{7}. 

23. A child that has not yet come to  the  use  of  reason
(\textjarman{may}) (\textjarman{may not}) receive Extreme Unction. 

24. A soldier about to  attack  the
enemy (\textjarman{may}) (\textjarman{may not}) receive Extreme Unction. 

25. A man who  has  swallowed
poison and is in danger of death (\textjarman{may}) (\textjarman{may not})  receive  Extreme  Unction.


26. Extreme Unction may be given as long as (\textjarman{apparent}) (\textjarman{real}) death has  not
occurred. 

27. In normal cases, real  death  does  not  occur  for  at  least
\spaces{7} after apparent death.  

28.  Extreme  Unction  can  be  repeated  in
\spaces{7} \spaces{7}  and  in  \spaces{7} \spaces{7}.  

29.  The  reception  of  this  sacrament   can
(\textjarman{sometimes}) (\textjarman{never}) be necessary for  salvation.  

30.  At  death,  a  person
should offer it as a \spaces{7} to God in union with the \spaces{7}  of  Christ,
in the \spaces{7}.


\newpage



1. At death we feel utterly alone, for each of us  is  a  \answer{person / self}.  
2.  The
priest can help the dying by the graces given in the sacrament of  \answer{Extreme Unction}.
3. St. James says: “Is any man \answer{sick} among you?  Let  him  bring  in  the
\answer{priests}, and let them \answer{pray} over  him,  \answer{anointing}  him  in  the  name  of
\answer{the Lord}. And the prayer of \answer{faith} shall save the sick man;  and  \answer{the Lord}
shall raise him up; and, if he be in \answer{sins}, they shall be \answer{forgiven}  him.”
4. He is here speaking  of  a  sign,  since  he  mentions  an  \answer{anointing}  and
\answer{prayer}. 
5. It is a sign of grace because it  gives  supernatural  help  to
the sick, and forgives \answer{sins}. 
6. It is an efficacious  sign,  since  this
promise is fulfilled \answer{by its use}. 
7. It is instituted by  Christ  since  it  is
administered in His \answer{Name} and with  His  \answer{authority}.  
8.  Hence,  it  is  a
\answer{sacrament} . 
9. Anglicans can see what the Catholic Church in  England  taught
concerning  Extreme  Unction,  before  the  Reformation,  by   reading   the
Sacerdotal Law of  \answer{Egbert},  Archbishop  of  \answer{York},  written  more  than
\answer{1200} years ago. 
10. The remote matter of Extreme  Unction  is  \answer{olive oil}.
11. Its proximate matter is \answer{the anointing}. 
12. The parts of  the  body  that  are
anointed are the \answer{Eye-lids, ears, nostrils,  lips,  hands  and  feet}. 
13. The long form  is  \answer{Through this  holy  anointing  and
His most tender mercy may God forgive  you  whatever  wrong  you  have  done
through sight. Amen.}.  
14.  The  short
form is \answer{Through this  holy
anointing, may God forgive thee whatever wrong thou hast done. Amen.}. 
15. This sacrament  (\answer{does})  (does  not)  give  sanctifying
grace. 
16. It forgives sin, provided the  recipient  has  \answer{contrition}  for  his
sins. 
17. It removes from the soul the remnants of sin, which are  \answer{weaknesses},
\answer{bad habits} and \answer{temporal punishment}. 
18. It brings supernatural \answer{peace} and  \answer{resignation}  to
the sick. 
19. It (\answer{can}) (cannot) make  the  soul  so  pure  that  it  can  go
straight to heaven. 
20.  It  (does)  (\answer{does  not})  regularly  restore  bodily
health by a miracle. 
21. It is foolish to delay too long in  having  Extreme
Unction administered because it  restores  bodily  health  if  \answer{God wills it},  if
\answer{the sacrament is received in good time} and if \answer{the recipient wants and believes}. 
22. It is because \answer{God knows best}, because  \answer{you can't rely on miracles}  and
because \answer{God never acts against our will}. 
23. A child that has not yet come to  the  use  of  reason
(may) (\answer{may not}) receive Extreme Unction. 
24. A soldier about to  attack  the
enemy (may) (\answer{may not}) receive Extreme Unction. 
25. A man who  has  swallowed
poison and is in danger of death (\answer{may}) (may not)  receive  Extreme  Unction.
26. Extreme Unction may be given as long as (apparent) (\answer{real}) death has  not
occurred. 
27. In normal cases, real  death  does  not  occur  for  at  least
\answer{half an  hour} after apparent death.  
28.  Extreme  Unction  can  be  repeated  in
\answer{a new sickness}  and  in  \answer{a new danger of the same sickness}.  
29.  The  reception  of  this  sacrament   can
(\answer{sometimes}) (never) be necessary for  salvation.  
30.  At  death,  a  person
should offer it as a \answer{sacrifice} to God in union with the \answer{offering}  of  Christ,
in the \answer{Mass}.

\end{document}

