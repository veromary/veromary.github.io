\documentclass{beamer}
\mode<presentation>
{
  \usetheme{Warsaw}
  % or ...

  \setbeamercovered{transparent}
  % or whatever (possibly just delete it)
}


\usepackage[english]{babel}
% or whatever

\usepackage[latin1]{inputenc}
% or whatever

\usepackage{times}
\usepackage[T1]{fontenc}
% Or whatever. Note that the encoding and the font should match. If T1
% does not look nice, try deleting the line with the fontenc.


\title[Living the Truth 53] % (optional, use only with long paper titles)
{Living the Truth: Chapter 53}

\subtitle
{Purgatory} % (optional)

\author{Rev.~C.~P.~Bowler S.M., M.A.}
% - Use the \inst{?} command only if the authors have different
%   affiliation.

%\subject{Talks}
% This is only inserted into the PDF information catalog. Can be left
% out. 



% If you have a file called "university-logo-filename.xxx", where xxx
% is a graphic format that can be processed by latex or pdflatex,
% resp., then you can add a logo as follows:

% \pgfdeclareimage[height=0.5cm]{university-logo}{university-logo-filename}
% \logo{\pgfuseimage{university-logo}}



% Delete this, if you do not want the table of contents to pop up at
% the beginning of each subsection:
\AtBeginSubsection[]
{
  \begin{frame}<beamer>{Outline}
    \tableofcontents[currentsection,currentsubsection]
  \end{frame}
}


% If you wish to uncover everything in a step-wise fashion, uncomment
% the following command: 

%\beamerdefaultoverlayspecification{<+->}


\begin{document}

\begin{frame}
  \titlepage
\end{frame}

\begin{frame}{Outline}
  \tableofcontents
  % You might wish to add the option [pausesections]
\end{frame}


% Since this a solution template for a generic talk, very little can
% be said about how it should be structured. However, the talk length
% of between 15min and 45min and the theme suggest that you stick to
% the following rules:  

% - Exactly two or three sections (other than the summary).
% - At *most* three subsections per section.
% - Talk about 30s to 2min per frame. So there should be between about
%   15 and 30 frames, all told.


\section{What}

\subsection{Definition}

\begin{frame}{Purgatory.}
\begin{itemize}
 \item Effects of sin:
\begin{itemize}
\item guilt,
\item punishment,
\item evil inclinations
\item and bad habits.
\end{itemize}
 \item How these are abolished.
 \item Need of purgatory.
\end{itemize}
\end{frame}

\begin{frame}{Definition of Purgatory}
We can therefore define \textbf{Purgatory} as a \textbf{place} and a \textbf{state} in which the  souls
of persons who die in \textbf{grace}  suffer  for  a  time  on  account  of  \textbf{temporal
punishment} due to sins committed after \textbf{Baptism}, before they are admitted  to
\textbf{Heaven}.
\end{frame}

\subsection{Evidence}

\begin{frame}{How We Know It Exists.}
\begin{itemize}
 \item Our act of divine faith.
 \item Role of the Church.
 \item Scripture and Tradition.
\end{itemize}
\end{frame}

\subsection{Fittingness}

\begin{frame}{Fittingness of Purgatory.}
\begin{itemize}
 \item Many not good enough for heaven, or bad enough for hell.
 \item Manifests God's sanctity, justice, and mercy.
 \item Most consoling and beneficial both to the living and the dead.
\end{itemize}
\end{frame}

\section{About}
\subsection{Sufferings}

\begin{frame}{Its Sufferings.}
\begin{itemize}
 \item Pain of loss. 
 \item Other sufferings.
 \item Material fire.
 \item Duration.
\end{itemize}
\end{frame}

\subsection{Joys}

\begin{frame}{Its Joys.}
\begin{itemize}
 \item Their source.
 \item Their nature.
\end{itemize}
\end{frame}

\section{Helps}

\subsection{Who}

\begin{frame}{Who Can Help The Holy Souls.}
\begin{itemize}
 \item Other souls in purgatory.
 \item Souls in heaven.
 \item Persons on earth.
\end{itemize}
\end{frame}

\subsection{How}

\begin{frame}{How We Can Help Them.}
\begin{itemize}
 \item Prayer.
 \item Good works, Alms-giving. Fasting.
 \item Our sufferings.
 \item The Mass.
 \item Indulgences.
 \item The Heroic Act.
\end{itemize}
\end{frame}

\subsection{Why}

\begin{frame}{Why We Ought to Help Them.}
\begin{itemize}
 \item Charity.
 \item Justice.
 \item Self-interest.
\end{itemize}
\end{frame}

\section{Resolutions}

\subsection{Avoid Purgatory}

\begin{frame}{How We Can Avoid Purgatory.}
\begin{itemize}
 \item Martyrdom.
 \item Baptism.
 \item Indulgences.
 \item The Mass.
 \item The Sacraments.
 \item The Religious Life.
 \item A spirit of penance.
 \item Sufferings.
\end{itemize}
\end{frame}

\subsection{Conclusions}

\begin{frame}{Practical Conclusions.}
\begin{itemize}
\item I should pray for the holy souls every time I pass a cemetery or  see  a
   funeral.


\item I should remember them whenever I give an alms.


\item I should make a daily intention to gain all indulgences I can.


\item I should take all the means I have of avoiding purgatory.


\item I should be truly devoted to the holy souls, since suffrages offered for
   me will benefit me in purgatory to the extent that I  have  helped  them
   during my life.
\end{itemize}
\end{frame}

\end{document}

