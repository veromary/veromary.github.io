% $Header: /Users/joseph/Documents/LaTeX/beamer/solutions/generic-talks/generic-ornate-15min-45min.en.tex,v 90e850259b8b 2007/01/28 20:48:30 tantau $

\documentclass{beamer}

% This file is a solution template for:

% - Giving a talk on some subject.
% - The talk is between 15min and 45min long.
% - Style is ornate.



% Copyright 2004 by Till Tantau <tantau@users.sourceforge.net>.
%
% In principle, this file can be redistributed and/or modified under
% the terms of the GNU Public License, version 2.
%
% However, this file is supposed to be a template to be modified
% for your own needs. For this reason, if you use this file as a
% template and not specifically distribute it as part of a another
% package/program, I grant the extra permission to freely copy and
% modify this file as you see fit and even to delete this copyright
% notice. 


\mode<presentation>
{
  \usetheme{Warsaw}
  % or ...

  \setbeamercovered{transparent}
  % or whatever (possibly just delete it)
}


\usepackage[english]{babel}
% or whatever

\usepackage[latin1]{inputenc}
% or whatever

\usepackage{times}
\usepackage[T1]{fontenc}
% Or whatever. Note that the encoding and the font should match. If T1
% does not look nice, try deleting the line with the fontenc.


\title[Living the Truth 51] % (optional, use only with long paper titles)
{Living the Truth: Chapter 51}

\subtitle
{Matrimony} % (optional)

\author{Rev.~C.~P.~Bowler S.M., M.A.}
% - Use the \inst{?} command only if the authors have different
%   affiliation.

%\subject{Talks}
% This is only inserted into the PDF information catalog. Can be left
% out. 



% If you have a file called "university-logo-filename.xxx", where xxx
% is a graphic format that can be processed by latex or pdflatex,
% resp., then you can add a logo as follows:

% \pgfdeclareimage[height=0.5cm]{university-logo}{university-logo-filename}
% \logo{\pgfuseimage{university-logo}}



% Delete this, if you do not want the table of contents to pop up at
% the beginning of each subsection:
\AtBeginSubsection[]
{
  \begin{frame}<beamer>{Outline}
    \tableofcontents[currentsection,currentsubsection]
  \end{frame}
}


% If you wish to uncover everything in a step-wise fashion, uncomment
% the following command: 

%\beamerdefaultoverlayspecification{<+->}


\begin{document}

\begin{frame}
  \titlepage
\end{frame}

\begin{frame}{Outline}
  \tableofcontents
  % You might wish to add the option [pausesections]
\end{frame}


% Since this a solution template for a generic talk, very little can
% be said about how it should be structured. However, the talk length
% of between 15min and 45min and the theme suggest that you stick to
% the following rules:  

% - Exactly two or three sections (other than the summary).
% - At *most* three subsections per section.
% - Talk about 30s to 2min per frame. So there should be between about
%   15 and 30 frames, all told.

\section{What is Matrimony}

\subsection{Definition}

\begin{frame}{What Matrimony Is.}
\begin{itemize}
 \item   Sex:     Expression of reverence and friendship.
         A sharing in God's creative act.
 \item   Regulated by marriage.
 \item   A contract:   Right given and accepted.
 \item    Why special:      Concerns persons. Established by God.
                  A sacrament, if both parties are baptised.
\end{itemize}
\end{frame}

\begin{frame}{Causes.}
\begin{itemize}
 \item   Efficient: Consent expressed externally.
 \item   Formal: Life-long bond.
 \item   Material: Man and woman.
 \item   Final:   
\begin{enumerate}
\item Generation and education of children.
\item     Benefits of the home life.
\end{enumerate}
\end{itemize}
\end{frame}

\subsection{Why a Sacrament}

\begin{frame}{Why a Sacrament.}
\begin{itemize}
 \item   Scripture. St. Paul. Union like that of Christ and the Church.
 \item   Tradition. Early Fathers.
 \item   Definition by Council of Trent.
\end{itemize}
\end{frame}

\begin{frame}{Analysis}
\begin{itemize}
 \item   Matter and Form: the consent.
 \item   Sign only: the consent. Thing and Sign: the bond as sacred.
 \item   Thing only: grace given all through life.
 \item   Ministers: the man and the woman.
\end{itemize}
\end{frame}

\begin{frame}{When a sacrament}
Marriage, as we  have  said,  is  necessarily  a  sacrament,  provided  \textbf{both}
parties have been baptised. If \textbf{only one} of them has been baptised, there  is
no sacrament.

\begin{itemize}
 \item   Why administered during Mass.
\end{itemize}
\end{frame}

\section{Properties.}

\subsection{Unity}

\begin{frame}{Unity.}
\begin{itemize}
 \item   Polyandry forbidden by the natural law. Why.
 \item   Polygamy forbidden by the natural law. Why.
\end{itemize}
\end{frame}

\subsection{Indissolubility}

\begin{frame}{Indissolubility.}
\begin{itemize}
 \item   Christ's teaching forbidding divorce.
 \item   The Church's life-long fight against divorce.
 \item   The Natural Law forbids any merely human power to  grant  divorce,
    because: 
\begin{itemize}
    \item It is opposed to the good of the children.
    \item It is opposed to the good of the husband and wife.
    \item It is opposed to the good of the state.
\end{itemize}
 \item  Separation allowed at times.
\end{itemize}
\end{frame}

\begin{frame}{Exceptions Made By God.}
\begin{itemize}
 \item   Marriage between two baptised.
\begin{itemize}
\item Papal dispensation
\item solemn religious profession
\end{itemize}
 \item   Marriage between two unbaptised.
\begin{itemize}
\item Papal dispensation
\item solemn religious profession
\item Pauline privilege
\end{itemize}
 \item   Marriage between a baptised and a non-baptised.
\begin{itemize}
\item Papal dispensation
\item solemn religious profession
\end{itemize}
\end{itemize}
\end{frame}

\section{Conditions}

\subsection{Impediments}

\begin{frame}{Impediments.}
\begin{itemize}
 \item   \textbf{Diriment.} (makes the marriage invalid)
 \item   \textbf{Prohibitive.} (makes the marriage illicit)
 \item   Dispensations.
 \item   Banns.
\end{itemize}
\end{frame}

\begin{frame}{Diriment Impediments}
\begin{itemize}
\item  \textbf{Age} --- males 16, females 14
\item  \textbf{A  Bond}  arising  from  a  previous  marriage. 
\item  \textbf{Diversity of Religion.} One of the parties is not baptised.
\item   \textbf{Holy Orders.}
\item  \textbf{Solemn Religious Profession.}
\item  \textbf{Consanguinity} or relationship by blood.
\item   \textbf{Affinity} or relationship by marriage.
\item \textbf{Spiritual Relationship} arising  from  baptism. 
\item  \textbf{Abduction.}
\item   \textbf{Legal adoption}
\end{itemize}
\end{frame}

\begin{frame}{Prohibitive Impediments}
Requires a dispensation, otherwise can make a marriage illicit (i.e. sinful.)
\begin{itemize}
\item Simple Vows
\item Mixed Marriage
\end{itemize}
\end{frame}


\subsection{Licit}

\begin{frame}{For Licity.}
\begin{itemize}
 \item   The parties must be in a state of grace.
 \item   They must be free from any impediments.
 \item   They must be sufficiently instructed in their religion.
 \item   They must observe the precepts laid down by the Church for the due
    celebration of marriage.
\end{itemize}
\end{frame}

\subsection{Valid}

\begin{frame}{For Validity.}
\begin{itemize}
  \item   The parties must be free from diriment impediments.


  \item   They must freely consent to the marriage.


  \item   If Catholics, they must normally  be  married  before  the  Parish
    Priest, or the Bishop, or a priest delegated by either  of  these;  and
    two witnesses.

\end{itemize}
\end{frame}

\section{Mixed Marriages}

\begin{frame}{Why The Church Forbids Mixed Marriages.}
\begin{itemize}
 \item   Fundamental division between husband and wife.
 \item   Danger of separation or divorce.
 \item   Danger of perversion.
 \item   Impossibility of rightly educating the children.
 \item   Dissension concerning vocations.
 \item   Dissension concerning moral matters.
\end{itemize}
\end{frame}

\section{Duties}

\begin{frame}{Duties of Parents Towards Children.}
\begin{itemize}
 \item   Children belong to them, not to the state.
 \item   Duty and right to educate them.
 \item   Why Catholic schools are necessary.
 \item   Need of home training. Value of a large family.
\end{itemize}
\end{frame}

\begin{frame}{Duties of Children Towards Parents.}
\begin{itemize}
 \item   Love, respect, obedience.
 \item   Foundations of these.
 \item   Example of Christ at Nazareth.
 \item   Need of self-denial and self-control.
\end{itemize}
\end{frame}


\end{document}

