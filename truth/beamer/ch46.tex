% $Header: /Users/joseph/Documents/LaTeX/beamer/solutions/generic-talks/generic-ornate-15min-45min.en.tex,v 90e850259b8b 2007/01/28 20:48:30 tantau $

\documentclass{beamer}

% This file is a solution template for:

% - Giving a talk on some subject.
% - The talk is between 15min and 45min long.
% - Style is ornate.



% Copyright 2004 by Till Tantau <tantau@users.sourceforge.net>.
%
% In principle, this file can be redistributed and/or modified under
% the terms of the GNU Public License, version 2.
%
% However, this file is supposed to be a template to be modified
% for your own needs. For this reason, if you use this file as a
% template and not specifically distribute it as part of a another
% package/program, I grant the extra permission to freely copy and
% modify this file as you see fit and even to delete this copyright
% notice. 


\mode<presentation>
{
  \usetheme{Warsaw}
  % or ...

  \setbeamercovered{transparent}
  % or whatever (possibly just delete it)
}


\usepackage[english]{babel}
% or whatever

\usepackage[latin1]{inputenc}
% or whatever

\usepackage{times}
\usepackage[T1]{fontenc}
% Or whatever. Note that the encoding and the font should match. If T1
% does not look nice, try deleting the line with the fontenc.


\title[Living the Truth 46] % (optional, use only with long paper titles)
{Living the Truth: Chapter 46}

\subtitle
{The Eucharist} % (optional)

\author{Rev.~C.~P.~Bowler S.M., M.A.}
% - Use the \inst{?} command only if the authors have different
%   affiliation.

%\subject{Talks}
% This is only inserted into the PDF information catalog. Can be left
% out. 



% If you have a file called "university-logo-filename.xxx", where xxx
% is a graphic format that can be processed by latex or pdflatex,
% resp., then you can add a logo as follows:

% \pgfdeclareimage[height=0.5cm]{university-logo}{university-logo-filename}
% \logo{\pgfuseimage{university-logo}}



% Delete this, if you do not want the table of contents to pop up at
% the beginning of each subsection:
\AtBeginSubsection[]
{
  \begin{frame}<beamer>{Outline}
    \tableofcontents[currentsection,currentsubsection]
  \end{frame}
}


% If you wish to uncover everything in a step-wise fashion, uncomment
% the following command: 

\beamerdefaultoverlayspecification{<+->}


\begin{document}

\begin{frame}
  \titlepage
\end{frame}

\begin{frame}{Outline}
  \tableofcontents
  % You might wish to add the option [pausesections]
\end{frame}


% Since this a solution template for a generic talk, very little can
% be said about how it should be structured. However, the talk length
% of between 15min and 45min and the theme suggest that you stick to
% the following rules:  

% - Exactly two or three sections (other than the summary).
% - At *most* three subsections per section.
% - Talk about 30s to 2min per frame. So there should be between about
%   15 and 30 frames, all told.

\section{The Eucharist}

\subsection{Looking at the Bible}

\begin{frame}{Christ's Promise}
\begin{itemize}
\item What it was.
\item How the Jews understood it.
\item How Christ approved of their interpretation.
\item How He re-acted to their refusal to accept it.
\end{itemize}
\end{frame}

\begin{frame}{Its Fulfilment}
\begin{itemize}
\item Christ's words at the Last Supper.
\item Why they must be taken in their proper sense.
\end{itemize}
\end{frame}

\begin{frame}{St. Paul's Testimony}
\begin{itemize}
\item In regard to unworthy Communions.
\item In regard to idolatry.
\end{itemize}
\end{frame}

\subsection{Looking at History}

\begin{frame}{Tradition}
\begin{itemize}
\item Early Documents.
\item The Fathers.
\item Drawings.
\item Inscriptions. 
\item Heretics.
\end{itemize}
\end{frame}

\subsection{Going in depth}

\begin{frame}{How He Becomes Present}
 
\begin{itemize}
\item Substance and accident.
\item Kinds of change: Accidental; substantial; transubstantiation.
\item Why transubstantiation is unique.
\end{itemize}
\end{frame}

\begin{frame}{Consequences}
\begin{itemize}
\item The accidents of bread and wine remain.
\item Body and Blood separated sacramentally.
\item Whole Christ under either species.
\item Christ present with His natural dimensions.
\item Christ present in every part of the Host.
\end{itemize}
\end{frame}

\begin{frame}{A Sacrament}
\begin{itemize}
\item Its matter and form.
\item Sign Only; Thing and Sign. Thing Only.
\item Definition of Trent.
\end{itemize}
\end{frame}

\begin{frame}{Its Worthy Reception}
\begin{itemize}
\item Baptised.
\item Grace.
\item Fasting.
\item Exceptions to the Fast.
\end{itemize}
\end{frame}

\subsection{Outcomes}

\begin{frame}{Its Effects}
\begin{itemize}
\item Union with Christ.
\item Union with Christ.
\item Union with fellowmen. 
\item Increase of grace.
\item Increase of charity. 
\item Increase of gifts. 
\item Supernatural nourishment. 
\item Assimilation to Christ.
\item Integrity.
\item Privilege of acting as host to God.
\item Glorious resurrection.
\end{itemize}
\end{frame}

\begin{frame}{Practical Conclusions.}
\begin{itemize}
\item I should receive Communion very often --- daily if I can.
\item I should make a good preparation and thanksgiving.
\item I should try to visit Our Lord in the Eucharist every day.
\item I should never miss Benediction when I can be present at it.
\item I should practise the devotion to the Eucharistic Heart of Christ.
\end{itemize}
\end{frame}
\end{document}


\section{Summary}

\begin{frame}{Outline}
  \tableofcontents
  % You might wish to add the option [pausesections]
\end{frame}



\end{document}


