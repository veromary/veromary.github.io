% $Header: /Users/joseph/Documents/LaTeX/beamer/solutions/generic-talks/generic-ornate-15min-45min.en.tex,v 90e850259b8b 2007/01/28 20:48:30 tantau $

\documentclass{beamer}

% This file is a solution template for:

% - Giving a talk on some subject.
% - The talk is between 15min and 45min long.
% - Style is ornate.



% Copyright 2004 by Till Tantau <tantau@users.sourceforge.net>.
%
% In principle, this file can be redistributed and/or modified under
% the terms of the GNU Public License, version 2.
%
% However, this file is supposed to be a template to be modified
% for your own needs. For this reason, if you use this file as a
% template and not specifically distribute it as part of a another
% package/program, I grant the extra permission to freely copy and
% modify this file as you see fit and even to delete this copyright
% notice. 


\mode<presentation>
{
  \usetheme{Warsaw}
  % or ...

  \setbeamercovered{transparent}
  % or whatever (possibly just delete it)
}


\usepackage[english]{babel}
% or whatever

\usepackage[latin1]{inputenc}
% or whatever

\usepackage{times}
\usepackage[T1]{fontenc}
% Or whatever. Note that the encoding and the font should match. If T1
% does not look nice, try deleting the line with the fontenc.


\title[Living the Truth 45] % (optional, use only with long paper titles)
{Living the Truth: Chapter 45}

\subtitle
{Confirmation} % (optional)

\author{Rev.~C.~P.~Bowler S.M., M.A.}
% - Use the \inst{?} command only if the authors have different
%   affiliation.

%\subject{Talks}
% This is only inserted into the PDF information catalog. Can be left
% out. 



% If you have a file called "university-logo-filename.xxx", where xxx
% is a graphic format that can be processed by latex or pdflatex,
% resp., then you can add a logo as follows:

% \pgfdeclareimage[height=0.5cm]{university-logo}{university-logo-filename}
% \logo{\pgfuseimage{university-logo}}



% Delete this, if you do not want the table of contents to pop up at
% the beginning of each subsection:
\AtBeginSubsection[]
{
  \begin{frame}<beamer>{Outline}
    \tableofcontents[currentsection,currentsubsection]
  \end{frame}
}


% If you wish to uncover everything in a step-wise fashion, uncomment
% the following command: 

\beamerdefaultoverlayspecification{<+->}


\begin{document}

\begin{frame}
  \titlepage
\end{frame}

\begin{frame}{Outline}
  \tableofcontents
  % You might wish to add the option [pausesections]
\end{frame}


% Since this a solution template for a generic talk, very little can
% be said about how it should be structured. However, the talk length
% of between 15min and 45min and the theme suggest that you stick to
% the following rules:  

% - Exactly two or three sections (other than the summary).
% - At *most* three subsections per section.
% - Talk about 30s to 2min per frame. So there should be between about
%   15 and 30 frames, all told.

\section{Confirmation}

\subsection{What Confirmation Is.}

\begin{frame}{Confirmation is a sacrament}
Confirmation is a sacrament that gives us light and strength to live as perfect \textbf{soliders of Christ}.
\end{frame}

\begin{frame}{Confirmation makes us adults}
It makes us adults in the supernatural order. See also Baptism.
\end{frame}

\subsection{How Christ Established It}

\begin{frame}{Not by administering it...}
\begin{itemize}
\item
Not by administering it,
\item
but by promising it.
\end{itemize}
\end{frame}

\begin{frame}{His promise to send the Holy Ghost}

After His Ascension...

\begin{quote}
It is expedient to you that I go; for, if I go, \textbf{I will send Him to you.}
\end{quote}

\end{frame}

\begin{frame}{Its elements as a sacrament}

\begin{itemize}
\item Sign only: the external ceremony
\item Thing and sign: the character
\item Thing only: the grace given
\end{itemize}

\end{frame}

\section{What it is made of}

\subsection{Matter and Form}

\begin{frame}{Remote Matter}

\begin{itemize}
\item Chrism, i.e. olive oil and balm.
\item Blessed by the Bishop.
\end{itemize}

\end{frame}

\begin{frame}{Proximate Matter}
\begin{itemize}
\item Imposition of hands.
\item The anointing.
\end{itemize}
\end{frame}

\begin{frame}{Form}
\begin{quote}
``I sign thee with the sign of the Cross, and I confirm thee with the chrism of salvation, in the name of the Father, and of the Son, and of the Holy Ghost.''
\end{quote}
\end{frame}

\subsection{Persons Concerned}

\begin{frame}{Minister}
\begin{itemize}
\item Ordinary
\item Extraordinary
\end{itemize}
\end{frame}

\begin{frame}{Subject}
\begin{itemize}
\item Baptized
\item Not already confirmed
\end{itemize}
\end{frame}

\begin{frame}{Sponsors}
\begin{itemize}
\item Need
\item Number
\item Qualifications
\item Obligations
\end{itemize}
\end{frame}

\section{Out to the world}

\subsection{Effects}

\begin{frame}{Immediate}
\begin{itemize}
\item The Character.
\item What it does for us.
\end{itemize}
\end{frame}

\begin{frame}{Ultimate}
\begin{itemize}
\item Graces of light and strength.
\item Gifts of the Holy Ghost.
\end{itemize}
\end{frame}

\subsection{Catholic Action}

\begin{frame}{Soldiers of Christ}
\begin{itemize}
\item
This sacrament makes us Apostles, soliders of Christ.
\item
With: \begin{itemize}
      \item breadth of outlook,
      \item a sense of responsibility,
      \item and a willingness to act for the common good.
\end{itemize}
\end{itemize}
\end{frame}

\begin{frame}{Scope given in Catholic Action}
\begin{itemize}
\item What it is.
\item Exhortations of the Popes.
\item Need for it today.
\item Requirements.
\item Greatness of the work.
\end{itemize}
\end{frame}

\section{Summary}

\begin{frame}{Outline}
  \tableofcontents
  % You might wish to add the option [pausesections]
\end{frame}



\end{document}


