% $Header: /Users/joseph/Documents/LaTeX/beamer/solutions/generic-talks/generic-ornate-15min-45min.en.tex,v 90e850259b8b 2007/01/28 20:48:30 tantau $

\documentclass{beamer}

% This file is a solution template for:

% - Giving a talk on some subject.
% - The talk is between 15min and 45min long.
% - Style is ornate.



% Copyright 2004 by Till Tantau <tantau@users.sourceforge.net>.
%
% In principle, this file can be redistributed and/or modified under
% the terms of the GNU Public License, version 2.
%
% However, this file is supposed to be a template to be modified
% for your own needs. For this reason, if you use this file as a
% template and not specifically distribute it as part of a another
% package/program, I grant the extra permission to freely copy and
% modify this file as you see fit and even to delete this copyright
% notice. 


\mode<presentation>
{
  \usetheme{Warsaw}
  % or ...

  \setbeamercovered{transparent}
  % or whatever (possibly just delete it)
}


\usepackage[english]{babel}
% or whatever

\usepackage[latin1]{inputenc}
% or whatever

\usepackage{times}
\usepackage[T1]{fontenc}
% Or whatever. Note that the encoding and the font should match. If T1
% does not look nice, try deleting the line with the fontenc.


\title[Living the Truth 48] % (optional, use only with long paper titles)
{Living the Truth: Chapter 48}

\subtitle
{The Liturgy of the Mass} % (optional)

\author{Rev.~C.~P.~Bowler S.M., M.A.}
% - Use the \inst{?} command only if the authors have different
%   affiliation.

%\subject{Talks}
% This is only inserted into the PDF information catalog. Can be left
% out. 



% If you have a file called "university-logo-filename.xxx", where xxx
% is a graphic format that can be processed by latex or pdflatex,
% resp., then you can add a logo as follows:

% \pgfdeclareimage[height=0.5cm]{university-logo}{university-logo-filename}
% \logo{\pgfuseimage{university-logo}}



% Delete this, if you do not want the table of contents to pop up at
% the beginning of each subsection:
\AtBeginSubsection[]
{
  \begin{frame}<beamer>{Outline}
    \tableofcontents[currentsection,currentsubsection]
  \end{frame}
}


% If you wish to uncover everything in a step-wise fashion, uncomment
% the following command: 

\beamerdefaultoverlayspecification{<+->}


\begin{document}

\begin{frame}
  \titlepage
\end{frame}

\begin{frame}{Outline}
  \tableofcontents
  % You might wish to add the option [pausesections]
\end{frame}


% Since this a solution template for a generic talk, very little can
% be said about how it should be structured. However, the talk length
% of between 15min and 45min and the theme suggest that you stick to
% the following rules:  

% - Exactly two or three sections (other than the summary).
% - At *most* three subsections per section.
% - Talk about 30s to 2min per frame. So there should be between about
%   15 and 30 frames, all told.

\section{Materials}

\subsection{Vestments}

\begin{frame}{The Vestments}
\begin{itemize}
\item Chasuble.
\item Maniple. 
\item Stole.
\item Alb. 
\item Amice.
\end{itemize}
\end{frame}

\begin{frame}{How they originated}
\begin{itemize}
\item Chasuble. - a cloak
\item Maniple. - a towel
\item Stole. - an handkerchief
\item Alb. - an inner garment
\item Amice. - a neck warmer
\end{itemize}
\end{frame}


\begin{frame}{What they symbolise}
\begin{itemize}
\item Chasuble. - Christ's yoke
\item Cincture - chastity
\item Alb. - purity
\end{itemize}
\end{frame}

\begin{frame}{Colours}
\begin{itemize}
\item white,
\item red,
\item green, 
\item violet
\item black.
\end{itemize}
\end{frame}

\subsection{Vessels}


\begin{frame}{Chalice}
\begin{itemize}
\item Chalice.
\item Paten;
\item Pall;
\item Veil;
\item Burse;
\item Corporal.
\end{itemize}
\end{frame}

\begin{frame}{Altar}
\begin{itemize}
\item Reminds us of a table and also of a tomb.
\end{itemize}
\end{frame}

\begin{frame}{Tabernacle}
\begin{itemize}
\item Ciborium
\item Cross,
\item Candles;
\item Cloths.
\item Altar stone.
\end{itemize}
\end{frame}

\begin{frame}{Side table}
Credence Table
\begin{itemize}
\item cruets,
\item dish and finger towel.
\end{itemize}
\end{frame}

\section{Form}

\subsection{Parts of the Mass}

\begin{frame}{Mass of the Catechumens}
\begin{itemize}
\item Sign of the Cross. 
\item Psalm  42.  
\item  Confiteor. 
\item  Introit. 
\item  Kyrie. 
\item Gloria. 
\end{itemize}
(continued next slide)
\end{frame}

\begin{frame}{Mass of the Catechumens (cont.)}
\begin{itemize}
\item Collect. 
\item Epistle and Gospel.  
\begin{itemize}
\item  Gradual,
\item  Alleluia,
\item Tract, 
\item Sequence. 
\end{itemize}
\item Creed.
\end{itemize}
\end{frame}

\begin{frame}{Mass of the Faithful}
\begin{itemize}
\item Offertory.
\item Consecration.
\item Communion.
\end{itemize}
\end{frame}

\section{Conclusion}

\begin{frame}{Moral}
Necessity of Living The Mass All Day and Every Day.
\end{frame}

\end{document}
