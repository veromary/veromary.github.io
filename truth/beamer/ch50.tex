% $Header: /Users/joseph/Documents/LaTeX/beamer/solutions/generic-talks/generic-ornate-15min-45min.en.tex,v 90e850259b8b 2007/01/28 20:48:30 tantau $

\documentclass{beamer}

% This file is a solution template for:

% - Giving a talk on some subject.
% - The talk is between 15min and 45min long.
% - Style is ornate.



% Copyright 2004 by Till Tantau <tantau@users.sourceforge.net>.
%
% In principle, this file can be redistributed and/or modified under
% the terms of the GNU Public License, version 2.
%
% However, this file is supposed to be a template to be modified
% for your own needs. For this reason, if you use this file as a
% template and not specifically distribute it as part of a another
% package/program, I grant the extra permission to freely copy and
% modify this file as you see fit and even to delete this copyright
% notice. 


\mode<presentation>
{
  \usetheme{Warsaw}
  % or ...

  \setbeamercovered{transparent}
  % or whatever (possibly just delete it)
}


\usepackage[english]{babel}
% or whatever

\usepackage[latin1]{inputenc}
% or whatever

\usepackage{times}
\usepackage[T1]{fontenc}
% Or whatever. Note that the encoding and the font should match. If T1
% does not look nice, try deleting the line with the fontenc.


\title[Living the Truth 50] % (optional, use only with long paper titles)
{Living the Truth: Chapter 50}

\subtitle
{Orders} % (optional)

\author{Rev.~C.~P.~Bowler S.M., M.A.}
% - Use the \inst{?} command only if the authors have different
%   affiliation.

%\subject{Talks}
% This is only inserted into the PDF information catalog. Can be left
% out. 



% If you have a file called "university-logo-filename.xxx", where xxx
% is a graphic format that can be processed by latex or pdflatex,
% resp., then you can add a logo as follows:

% \pgfdeclareimage[height=0.5cm]{university-logo}{university-logo-filename}
% \logo{\pgfuseimage{university-logo}}



% Delete this, if you do not want the table of contents to pop up at
% the beginning of each subsection:
\AtBeginSubsection[]
{
  \begin{frame}<beamer>{Outline}
    \tableofcontents[currentsection,currentsubsection]
  \end{frame}
}


% If you wish to uncover everything in a step-wise fashion, uncomment
% the following command: 

%\beamerdefaultoverlayspecification{<+->}


\begin{document}

\begin{frame}
  \titlepage
\end{frame}

\begin{frame}{Outline}
  \tableofcontents
  % You might wish to add the option [pausesections]
\end{frame}


% Since this a solution template for a generic talk, very little can
% be said about how it should be structured. However, the talk length
% of between 15min and 45min and the theme suggest that you stick to
% the following rules:  

% - Exactly two or three sections (other than the summary).
% - At *most* three subsections per section.
% - Talk about 30s to 2min per frame. So there should be between about
%   15 and 30 frames, all told.

\section{Definition}

\subsection{Social}

\begin{frame}{Two Social Sacraments}
\begin{itemize}
\item Orders.
\item Matrimony.
\end{itemize}
\end{frame}

\subsection{Sacrament}

\begin{frame}{Orders is a Sacrament}
\begin{itemize}
 \item Institution of the Priesthood at the Last Supper.
 \item Practice of the Apostles concerning ordination.
 \item Teaching of the Council of Trent.
\end{itemize}
\end{frame}

\subsection{Major/Minor}

\begin{frame}{Minor and Major Orders}
\begin{itemize}
 \item Minor: Door-keeper; Reader; Exorcist; Acolyte.
 \item Major: Sub-deaconate, Deaconate; Priesthood; Episcopate.
\end{itemize}
\end{frame}

\subsection{Matter and Form}

\begin{frame}{Matter and Form}
\begin{itemize}
 \item Matter: Imposition of hands for the Major Orders.
 \item Form: Certain words found in the Preface.
\end{itemize}
``Bestow, we beseech Thee,  Father  Almighty,
on these Thy servants, the dignity of the priesthood. Renew in their  hearts
the Spirit of holiness, that they may receive from Thee,  O  God,  and  hold
the office of second rank, and command by the example of their own  lives  a
strict standard of morals.''
\end{frame}

\subsection{Excellence}

\begin{frame}{Dignity of the Priesthood}
\begin{itemize}
 \item    Christ's Priesthood: the most excellent possible --- Union with God, Victim, people.
 \item    Dignity of Catholic Priesthood:
\begin{itemize}
      \item Sharing in Christ's.
      \item Consecration at Mass.
      \item Absolution.
      \item Power over the Eucharist.
      \item Sacramental Graces.
\end{itemize}
\end{itemize}
\end{frame}

\section{Who gets ordained?}

\subsection{Valid/Licit}

\begin{frame}{Conditions for Receiving the Priesthood}
\begin{itemize}
 \item    Validity: Baptised. A Male. Intention.
 \item    Licity: Grace; confirmation; age; knowledge; virtue; reception  of inferior orders, etc.
\end{itemize}
\end{frame}

\subsection{Calling}

\begin{frame}{Vocation to the Priesthood}
\begin{itemize}
 \item Invitation to receive ordination --- made by God --- through a bishop.
 \item Certitude regarding it.
\end{itemize}
\end{frame}

\subsection{Who gets in?}

\begin{frame}{Conditions for Entering a Seminary}
\begin{itemize}
 \item Right Intention.
 \item Physical Fitness.
 \item Intellectual Fitness.
 \item Moral Fitness.
\end{itemize}
\end{frame}

\subsection{Religious life}

\begin{frame}{The Religious Life}
\begin{itemize}
 \item Vows of Poverty, Chastity and Obedience.
 \item Remove obstacles to perfection.
\end{itemize}
\end{frame}

\section{Conclusion}

\subsection{How to grow your Own Seminarian}

\begin{frame}{Home-Life and Vocations.}
In such a home, the children see in their parents a model of an upright,
  industrious and pious life. They see their parents loving each  other  in
  Our Lord, see them approach the holy sacraments frequently, and not  only
  obey the laws of the Church concerning abstinence and fasting,  but  also
  observe the spirit of voluntary Christian mortification.  They  see  them
  pray at home, gathering around them all the family,  that  common  prayer
  may rise more acceptably to heaven. They find them compassionate  towards
  the sufferings of others, and see them divide with the poor the  much  or
  the little they possess. --- Pius XI
\end{frame}

\subsection{Vatican II adjustments}

\begin{frame}{Editor's note}
  Cf.  Ministeria  Quaedam;  The  Second  Vatican  Council
changed the structure of Minor Orders: ``Among the special offices which  are
to be retained and adapted to present-day needs there  are  some  which  are
essentially connected with the ministries of the word and of the  altar.  In
the Latin Church they are the office of lector, the office  of  acolyte  and
the subdiaconate. THESE OFFICES WILL NOW BE REDUCED TO TWO, THAT  OF  LECTOR
AND THAT OF ACOLYTE, AND THE FUNCTIONS OF THE SUBDIACONATE WILL  BE  DIVIDED
BETWEEN THEM \ldots\ It is in keeping with  the  nature  of  the  case  and  with
contemporary attitudes that such  ministries  should  no  longer  be  called
`minor orders.' Their conferring will no longer be called  `ORDINATION'  but
`INSTALLATION'. ''
[1] [Editor's note]. Cf. Canon 1031.
\end{frame}

\end{document}
