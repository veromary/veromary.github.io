\documentclass{beamer}
\mode<presentation>
{
  \usetheme{Warsaw}
  % or ...

  \setbeamercovered{transparent}
  % or whatever (possibly just delete it)
}


\usepackage[english]{babel}
% or whatever

\usepackage[latin1]{inputenc}
% or whatever

\usepackage{times}
\usepackage[T1]{fontenc}
% Or whatever. Note that the encoding and the font should match. If T1
% does not look nice, try deleting the line with the fontenc.


\title[Living the Truth 55] % (optional, use only with long paper titles)
{Living the Truth: Chapter 55}

\subtitle
{The End of the World} % (optional)

\author{Rev.~C.~P.~Bowler S.M., M.A.}
% - Use the \inst{?} command only if the authors have different
%   affiliation.

%\subject{Talks}
% This is only inserted into the PDF information catalog. Can be left
% out. 



% If you have a file called "university-logo-filename.xxx", where xxx
% is a graphic format that can be processed by latex or pdflatex,
% resp., then you can add a logo as follows:

% \pgfdeclareimage[height=0.5cm]{university-logo}{university-logo-filename}
% \logo{\pgfuseimage{university-logo}}



% Delete this, if you do not want the table of contents to pop up at
% the beginning of each subsection:
\AtBeginSubsection[]
{
  \begin{frame}<beamer>{Outline}
    \tableofcontents[currentsection,currentsubsection]
  \end{frame}
}


% If you wish to uncover everything in a step-wise fashion, uncomment
% the following command: 

%\beamerdefaultoverlayspecification{<+->}


\begin{document}

\begin{frame}
  \titlepage
\end{frame}

\begin{frame}{Outline}
  \tableofcontents
  % You might wish to add the option [pausesections]
\end{frame}


% Since this a solution template for a generic talk, very little can
% be said about how it should be structured. However, the talk length
% of between 15min and 45min and the theme suggest that you stick to
% the following rules:  

% - Exactly two or three sections (other than the summary).
% - At *most* three subsections per section.
% - Talk about 30s to 2min per frame. So there should be between about
%   15 and 30 frames, all told.


\section{Destruction of the World.}

\subsection{The Fact}

\begin{frame}{The Fact.}
\begin{itemize}
\item When.
\item How.
\item Features.
\end{itemize}
\end{frame}
\subsection{Signs}

\begin{frame}{Signs.}
\begin{itemize}
\item Gospel preached to all.
\item Coming of Anti-Christ.
\item Return of Elias.
\item Conversion of the Jews.
\end{itemize}
\end{frame}

\section{Resurrection.}

\subsection{What}

\begin{frame}{What.}
\begin{itemize}
\item Substantial reunion of body and soul.
\item Supernatural.
\item Indissoluble.
\end{itemize}
\end{frame}

\subsection{Christ's Resurrection}

\begin{frame}{Christ's Resurrection.}
\begin{itemize}
\item He raised the dead to life.
\item He foretold His own resurrection.
\item He rose from the dead.
\item Why we believe it.
\end{itemize}
\end{frame}

\subsection{General Resurrection}

\begin{frame}{General Resurrection.}
\begin{itemize}
\item Teaching of Christ.
\item Teaching of the Apostles.
\item Tradition.
\end{itemize}
\end{frame}

\subsection{The Glorified Body}

\begin{frame}{Features of a Glorified Body.}
\begin{itemize}
\item Subtlety.
\item Agility.
\item Incorruptibility.
\item Clarity.
\end{itemize}
\end{frame}

\section{Last Judgement.}

\begin{frame}{Its Nature and Purpose.}
\begin{itemize}
\item Man as a member of the race.
\item Vindication of Providence.
\item The last Sentence given by the Judge.
\item The value of fraternal charity.
\end{itemize}
\end{frame}

\begin{frame}{The Need To Be Ever Ready For It.}
\begin{center}
``ABIDE IN HIM.''
\end{center}
\end{frame}

\end{document}
 
