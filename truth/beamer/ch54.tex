\documentclass{beamer}
\mode<presentation>
{
  \usetheme{metropolis}
  % or ...

  \setbeamercovered{transparent}
  % or whatever (possibly just delete it)
}


\usepackage[english]{babel}
% or whatever

\usepackage[latin1]{inputenc}
% or whatever

\usepackage{times}
\usepackage[T1]{fontenc}
% Or whatever. Note that the encoding and the font should match. If T1
% does not look nice, try deleting the line with the fontenc.


\title[Living the Truth 54] % (optional, use only with long paper titles)
{Living the Truth: Chapter 54}

\subtitle
{Heaven and Hell} % (optional)

\author{Rev.~C.~P.~Bowler S.M., M.A.}
% - Use the \inst{?} command only if the authors have different
%   affiliation.

%\subject{Talks}
% This is only inserted into the PDF information catalog. Can be left
% out. 



% If you have a file called "university-logo-filename.xxx", where xxx
% is a graphic format that can be processed by latex or pdflatex,
% resp., then you can add a logo as follows:

% \pgfdeclareimage[height=0.5cm]{university-logo}{university-logo-filename}
% \logo{\pgfuseimage{university-logo}}



% Delete this, if you do not want the table of contents to pop up at
% the beginning of each subsection:
\AtBeginSubsection[]
{
  \begin{frame}<beamer>{Outline}
    \tableofcontents[currentsection,currentsubsection]
  \end{frame}
}


% If you wish to uncover everything in a step-wise fashion, uncomment
% the following command: 

%\beamerdefaultoverlayspecification{<+->}


\begin{document}

\begin{frame}
  \titlepage
\end{frame}

\begin{frame}{Outline}
  \tableofcontents
  % You might wish to add the option [pausesections]
\end{frame}


% Since this a solution template for a generic talk, very little can
% be said about how it should be structured. However, the talk length
% of between 15min and 45min and the theme suggest that you stick to
% the following rules:  

% - Exactly two or three sections (other than the summary).
% - At *most* three subsections per section.
% - Talk about 30s to 2min per frame. So there should be between about
%   15 and 30 frames, all told.


\section{Introduction.}

\subsection{Our Last End}

\begin{frame}{OUR LAST END}
\begin{itemize}
\item   To promote God's glory in eternity.
\item   Our choice of heaven or hell.
\end{itemize}
\end{frame}

\section{Heaven.}

\subsection{Man's Greatest Needs}
\begin{frame}{Man's Greatest Needs.}
\begin{itemize}
\item   Knowledge. How the blessed know God in heaven.
\item   Love. How they love Him in heaven.
\item   Pleasure. Sensible delights, after the resurrection.
\end{itemize}
\end{frame}
\subsection{What Heaven Is}
\begin{frame}{What Heaven Is.}
\begin{itemize}
 \item  Why called the beatific vision.
 \item  Its definition.
\end{itemize}
\end{frame}
\subsection{Its Properties}
\begin{frame}{Its Properties.}
\begin{itemize}
 \item  It is unlosable.
 \item  It is eternal.
 \item  It is ever new.
 \item  It varies for different persons.
 \item  Yet each is perfectly happy.
\end{itemize}
\end{frame}
\begin{frame}{Conclusion.}
\begin{itemize}
 \item  Eternal life.
 \item  The one thing we must never lose.
 \item  Our home in eternity.
\end{itemize}
\end{frame}

\section{Hell.}

\subsection{Its Origin}
\begin{frame}{Its Origin.}
\begin{itemize}
 \item  The angels.
 \item  Their fall.
\end{itemize}
\end{frame}
\subsection{How we know}
\begin{frame}{Scripture And Tradition.}
\begin{itemize}
 \item  Christ's Teaching.
 \item  The Teaching of the Apostles.
 \item  Tradition.
\end{itemize}
\end{frame}
\subsection{Its Sufferings}
\begin{frame}{Its Sufferings.}
\begin{itemize}
 \item  The pain of loss.
 \item  Despair.
 \item  Remorse.
 \item  Fire.
\end{itemize}
\end{frame}
\subsection{Its Eternity}
\begin{frame}{Its Eternity.}
\begin{itemize}
 \item  Definition of the Church.
 \item  Why hell is eternal.
 \item  Importance of death.
\end{itemize}
\end{frame}

\section{Conclusion}

\subsection{Lessons}

\begin{frame}{Lessons.}
\begin{itemize}
 \item  Dignity and depth of the soul.
 \item  Gratitude for the Incarnation.
 \item  Hatred of sin.
 \item  Value of the present moment.
 \item  Danger of worldliness and superficiality.
 \item  Zeal for souls.
 \item  Imitation of Christ and Our Lady.
\end{itemize}
\end{frame}
\begin{frame}{Practical Conclusions.}

\begin{itemize}
\item  I should hate mortal sin above every other evil; and avoid as far as  I
  can all occasions of sin.


\item  I should make an act of  perfect  contrition  immediately,  and  go  to
  confession as soon as I can if I ever fall into mortal sin.


\item  I should do my best to increase good habits daily,  by  practising  all
  the virtues.


\item  I should be deeply devoted to Our Lady, since this is  a  guarantee  of
  salvation.


\item  I should never forget Our Lord's words: ``What shall it profit a man  if
  he gain the whole world, and suffer the loss of his soul?''
\end{itemize}
\end{frame}

\end{document}


