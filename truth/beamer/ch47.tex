% $Header: /Users/joseph/Documents/LaTeX/beamer/solutions/generic-talks/generic-ornate-15min-45min.en.tex,v 90e850259b8b 2007/01/28 20:48:30 tantau $

\documentclass{beamer}

% This file is a solution template for:

% - Giving a talk on some subject.
% - The talk is between 15min and 45min long.
% - Style is ornate.



% Copyright 2004 by Till Tantau <tantau@users.sourceforge.net>.
%
% In principle, this file can be redistributed and/or modified under
% the terms of the GNU Public License, version 2.
%
% However, this file is supposed to be a template to be modified
% for your own needs. For this reason, if you use this file as a
% template and not specifically distribute it as part of a another
% package/program, I grant the extra permission to freely copy and
% modify this file as you see fit and even to delete this copyright
% notice. 


\mode<presentation>
{
  \usetheme{Warsaw}
  % or ...

  \setbeamercovered{transparent}
  % or whatever (possibly just delete it)
}


\usepackage[english]{babel}
% or whatever

\usepackage[latin1]{inputenc}
% or whatever

\usepackage{times}
\usepackage[T1]{fontenc}
% Or whatever. Note that the encoding and the font should match. If T1
% does not look nice, try deleting the line with the fontenc.


\title[Living the Truth 47] % (optional, use only with long paper titles)
{Living the Truth: Chapter 47}

\subtitle
{The Mass as a Sacrifice} % (optional)

\author{Rev.~C.~P.~Bowler S.M., M.A.}
% - Use the \inst{?} command only if the authors have different
%   affiliation.

%\subject{Talks}
% This is only inserted into the PDF information catalog. Can be left
% out. 



% If you have a file called "university-logo-filename.xxx", where xxx
% is a graphic format that can be processed by latex or pdflatex,
% resp., then you can add a logo as follows:

% \pgfdeclareimage[height=0.5cm]{university-logo}{university-logo-filename}
% \logo{\pgfuseimage{university-logo}}



% Delete this, if you do not want the table of contents to pop up at
% the beginning of each subsection:
\AtBeginSubsection[]
{
  \begin{frame}<beamer>{Outline}
    \tableofcontents[currentsection,currentsubsection]
  \end{frame}
}


% If you wish to uncover everything in a step-wise fashion, uncomment
% the following command: 

\beamerdefaultoverlayspecification{<+->}


\begin{document}

\begin{frame}
  \titlepage
\end{frame}

\begin{frame}{Outline}
  \tableofcontents
  % You might wish to add the option [pausesections]
\end{frame}


% Since this a solution template for a generic talk, very little can
% be said about how it should be structured. However, the talk length
% of between 15min and 45min and the theme suggest that you stick to
% the following rules:  

% - Exactly two or three sections (other than the summary).
% - At *most* three subsections per section.
% - Talk about 30s to 2min per frame. So there should be between about
%   15 and 30 frames, all told.

\section{The Mass as a Sacrifice}

\subsection{Pre-eminence}

\begin{frame}{Our Primary Duty}
\begin{itemize}
\item To pay our debt of worship to God as our \textbf{Creator}, \textbf{Conserver} and  
\textbf{Last End}.
\item By an external act which expresses our homage.
\end{itemize}
\end{frame}


\begin{frame}{What is a Sacrifice}
\begin{itemize}
\item An act of homage due to God alone, 
\item since it honours Him  precisely    as Creator.
\end{itemize}
\end{frame}

\begin{frame}{Sacrifice is Natural}
\begin{itemize}
\item Natural to man. 
\item Universally practised even by peoples outside  the
    Chosen Race.
\end{itemize}
\end{frame}

\begin{frame}{Sacrifice Implies:}
\begin{itemize}
\item  Internal dispositions, especially those of reverence and  complete
       subjection to God.
\item  Expressed externally by the oblation and immolation of a victim.
\item Performed by a priest on behalf of the community he represents.
\end{itemize}
\end{frame}

\begin{frame}{The Central Sacrifice.}
\begin{itemize}
\item  The Sacrifice of Calvary.
\item   Why central.
\item   Why all perfect and infinitely pleasing to God.
\end{itemize}
\end{frame}

\subsection{Last Supper, Mass and Calvary}

\begin{frame}{The Last Supper}
\begin{itemize}
\item    The double consecration of bread and wine.
\item    How this gives us Calvary by freeing it from the  limits  of  time
   and place.
\item    Christ's command: ``Do this in commemoration of Me.''
\end{itemize}
\end{frame}

\begin{frame}{The Mass}
\begin{itemize}
\item    In virtue of the character given  him  at  ordination  the  priest
    today does what Christ did in the Supper Room.
\item    He puts Christ's death present before us in such a way that we can
    all take an active part in His Sacrifice by offering  it  with  Him  to
    God.
\item    By the Mass, then, we are able to give God  a  worship  worthy  of
    Him.
\end{itemize}
\end{frame}

\begin{frame}{The Mass and Calvary}
\begin{itemize}
\item    The Mass is substantially the same sacrifice as that of the Cross.
\item    How they are the same, and how they differ:
\begin{itemize}
      \item The efficient cause in each --- the Priest
     \item The material cause in each --- the Victim
    \item The formal cause in each --- the Immolation
     \item The final cause in each --- the worship due to God
\end{itemize}
\end{itemize}
\end{frame}

\begin{frame}{Our Part in the Mass}
\begin{itemize}
\item    The Mass is our sacrifice as well as Christ's.
\item    We should take a very active part in it by offering  it  in  union
    with Christ and with the priest at the altar who represents us.
\item    This implies an offering of ourselves to God.
\item    It implies too, dispositions of  subjection,  reverence  and  love
    like those that filled Christ's soul on Calvary.
\item    We should strive to live the Mass daily by increasing  these  good
    dispositions.
\end{itemize}
\end{frame}

\begin{frame}{The Value of the Mass}
\begin{itemize}
\item It enables us to offer to God a sacrifice of infinite worth.
\item It brings down God's gifts to us:
\begin{itemize}
      \item It forgives sin by the actual graces it offers.
     \item It remits temporal punishment.
    \item It obtains spiritual and temporal goods for us.
     \item At Mass we can receive Holy Communion.
\end{itemize}
\item The Mass is the very centre of Christian worship.
\item How we should treasure it as our Saviour's greatest gift.
\item How we should live it in our daily life.
\end{itemize}
\end{frame}

\begin{frame}{Practical Conclusions}
\begin{enumerate}
\item   I should never miss Mass when I can reasonably be expected to go.
\item   I should take a very active part  in  the  Mass,  and  not  be  just  a
  passive spectator.
\item   I should always be in time for Mass, and remain there to the end.
\item   I should receive Holy Communion, as far as possible, at  every  Mass  I
  attend.
\item   I should strive to live the Mass daily by avoiding sin  and  conforming
  my will ever more and more perfectly to God's.
\end{enumerate}
\end{frame}


\end{document}
