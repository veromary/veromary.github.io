% $Header: /Users/joseph/Documents/LaTeX/beamer/solutions/generic-talks/generic-ornate-15min-45min.en.tex,v 90e850259b8b 2007/01/28 20:48:30 tantau $

\documentclass{beamer}

% This file is a solution template for:

% - Giving a talk on some subject.
% - The talk is between 15min and 45min long.
% - Style is ornate.



% Copyright 2004 by Till Tantau <tantau@users.sourceforge.net>.
%
% In principle, this file can be redistributed and/or modified under
% the terms of the GNU Public License, version 2.
%
% However, this file is supposed to be a template to be modified
% for your own needs. For this reason, if you use this file as a
% template and not specifically distribute it as part of a another
% package/program, I grant the extra permission to freely copy and
% modify this file as you see fit and even to delete this copyright
% notice. 


\mode<presentation>
{
  \usetheme{Warsaw}
  % or ...

  \setbeamercovered{transparent}
  % or whatever (possibly just delete it)
}


\usepackage[english]{babel}
% or whatever

\usepackage[latin1]{inputenc}
% or whatever

\usepackage{times}
\usepackage[T1]{fontenc}
% Or whatever. Note that the encoding and the font should match. If T1
% does not look nice, try deleting the line with the fontenc.


\title[Living the Truth 52] % (optional, use only with long paper titles)
{Living the Truth: Chapter 52}

\subtitle
{Death and Judgement} % (optional)

\author{Rev.~C.~P.~Bowler S.M., M.A.}
% - Use the \inst{?} command only if the authors have different
%   affiliation.

%\subject{Talks}
% This is only inserted into the PDF information catalog. Can be left
% out. 



% If you have a file called "university-logo-filename.xxx", where xxx
% is a graphic format that can be processed by latex or pdflatex,
% resp., then you can add a logo as follows:

% \pgfdeclareimage[height=0.5cm]{university-logo}{university-logo-filename}
% \logo{\pgfuseimage{university-logo}}



% Delete this, if you do not want the table of contents to pop up at
% the beginning of each subsection:
\AtBeginSubsection[]
{
  \begin{frame}<beamer>{Outline}
    \tableofcontents[currentsection,currentsubsection]
  \end{frame}
}


% If you wish to uncover everything in a step-wise fashion, uncomment
% the following command: 

%\beamerdefaultoverlayspecification{<+->}


\begin{document}

\begin{frame}
  \titlepage
\end{frame}

\begin{frame}{Outline}
  \tableofcontents
  % You might wish to add the option [pausesections]
\end{frame}


% Since this a solution template for a generic talk, very little can
% be said about how it should be structured. However, the talk length
% of between 15min and 45min and the theme suggest that you stick to
% the following rules:  

% - Exactly two or three sections (other than the summary).
% - At *most* three subsections per section.
% - Talk about 30s to 2min per frame. So there should be between about
%   15 and 30 frames, all told.

\section{What is Death}

\subsection{God warns us}

\begin{frame}{God's Warning.}
\begin{itemize}
 \item    He tells me that death is inevitable.
 \item    He warns me to be always ready to die.
 \item    He tells me what death is: the dissolution of the human composite.
\end{itemize}
\end{frame}

\subsection{Kinds}

\begin{frame}{Kinds of Death.}
\begin{itemize}
 \item    Final perseverance --- death in a state of grace --- a happy death.
 \item    Final impenitence --- death in personal mortal sin --- a bad death.
 \item    The folly of putting off conversion.
 \item    Examples of death-bed repentance.
\end{itemize}
\end{frame}

\subsection{Effects}

\begin{frame}{Effects of Death. }
\begin{itemize}
 \item    It strips from us all worldly goods forever.
 \item    It ends forever our chance of merit.
 \item    It brings before us the dread alternative: heaven or hell forever.
\end{itemize}
\end{frame}

\section{Judgement}

\subsection{Particular/General}

\begin{frame}{The First Judgment.}
\begin{itemize}
 \item    The Particular Judgement. The General Judgement. Need for each.
 \item    It implies: examination, sentence, execution.
 \item    How each is carried out; and where.
\end{itemize}
\end{frame}

\section{Views of Death}

\subsection{Worldly}

\begin{frame}{The Worldling's View of Death.}
\begin{itemize}
 \item    The greatest evil.
 \item    Refusal to think about it.
 \item    Fear of it poisons happiness.
\end{itemize}
\end{frame}

\subsection{Catholic}

\begin{frame}{The Catholic's View of Death.}
\begin{itemize}
 \item    A going home.
 \item    A birthday into heaven.
 \item    God's last grace to us on earth.
\end{itemize}
\end{frame}

\begin{frame}{Means He has left us to die well}
\begin{itemize}
\item Last Confession.
\item Last Communion.
\item   Extreme Unction. 
\item Last Blessing.
\end{itemize}
\end{frame}

\section{Conclusion}

\begin{frame}{Practical Conclusions.}
\begin{enumerate}
\item  I should do now what I shall wish to have done at death.
\item  I should often pray for the dying.
\item  I should love the Hail Mary, since in it I ask for Our Lady  to  obtain
  for me the grace of a happy death.
\item  I should also have Masses said for this great grace.
\item  I should hate sin with all  my  heart,  since  it  alone  can  make  me
  miserable at death.
\end{enumerate}
\end{frame}

\end{document}
