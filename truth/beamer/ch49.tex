% $Header: /Users/joseph/Documents/LaTeX/beamer/solutions/generic-talks/generic-ornate-15min-45min.en.tex,v 90e850259b8b 2007/01/28 20:48:30 tantau $

\documentclass{beamer}

% This file is a solution template for:

% - Giving a talk on some subject.
% - The talk is between 15min and 45min long.
% - Style is ornate.



% Copyright 2004 by Till Tantau <tantau@users.sourceforge.net>.
%
% In principle, this file can be redistributed and/or modified under
% the terms of the GNU Public License, version 2.
%
% However, this file is supposed to be a template to be modified
% for your own needs. For this reason, if you use this file as a
% template and not specifically distribute it as part of a another
% package/program, I grant the extra permission to freely copy and
% modify this file as you see fit and even to delete this copyright
% notice. 


\mode<presentation>
{
  \usetheme{Warsaw}
  % or ...

  \setbeamercovered{transparent}
  % or whatever (possibly just delete it)
}


\usepackage[english]{babel}
% or whatever

\usepackage[latin1]{inputenc}
% or whatever

\usepackage{times}
\usepackage[T1]{fontenc}
% Or whatever. Note that the encoding and the font should match. If T1
% does not look nice, try deleting the line with the fontenc.


\title[Living the Truth 49] % (optional, use only with long paper titles)
{Living the Truth: Chapter 49}

\subtitle
{Extreme Unction} % (optional)

\author{Rev.~C.~P.~Bowler S.M., M.A.}
% - Use the \inst{?} command only if the authors have different
%   affiliation.

%\subject{Talks}
% This is only inserted into the PDF information catalog. Can be left
% out. 



% If you have a file called "university-logo-filename.xxx", where xxx
% is a graphic format that can be processed by latex or pdflatex,
% resp., then you can add a logo as follows:

% \pgfdeclareimage[height=0.5cm]{university-logo}{university-logo-filename}
% \logo{\pgfuseimage{university-logo}}



% Delete this, if you do not want the table of contents to pop up at
% the beginning of each subsection:
\AtBeginSubsection[]
{
  \begin{frame}<beamer>{Outline}
    \tableofcontents[currentsection,currentsubsection]
  \end{frame}
}


% If you wish to uncover everything in a step-wise fashion, uncomment
% the following command: 

%\beamerdefaultoverlayspecification{<+->}


\begin{document}

\begin{frame}
  \titlepage
\end{frame}

\begin{frame}{Outline}
  \tableofcontents
  % You might wish to add the option [pausesections]
\end{frame}


% Since this a solution template for a generic talk, very little can
% be said about how it should be structured. However, the talk length
% of between 15min and 45min and the theme suggest that you stick to
% the following rules:  

% - Exactly two or three sections (other than the summary).
% - At *most* three subsections per section.
% - Talk about 30s to 2min per frame. So there should be between about
%   15 and 30 frames, all told.

\section{Definition}

\subsection{What Extreme Unction is}

\begin{frame}{Special need at death}
\begin{itemize}
\item Anointing
\item Prayer
\item health of soul
\item health of body
\end{itemize}
\end{frame}

\begin{frame}{Sacrament}
\begin{itemize}
 \item Scripture.
 \item Tradition.
 \item Definition of Trent.
\end{itemize}
\end{frame}

\subsection{Matter and Form.}

\begin{frame}{Matter}
\begin{itemize}
\item Remote:  olive oil specially blessed.
\item Proximate: The anointing of the senses.
\begin{itemize}
\item Eye-lids,
\item ears, 
\item nostrils,
\item  lips,
\item  hands 
\item and  feet 
\end{itemize}
\end{itemize}
\end{frame}

\begin{frame}{Form}
\begin{itemize}
\item Long,
\emph{Through this  holy  anointing  and
His most tender mercy may God forgive  you  whatever  wrong  you  have  done
through} (insert sense here) \emph{. Amen.}
\item Short.
\emph{Through this  holy
anointing, may God forgive thee whatever wrong thou hast done. Amen.}
\end{itemize}
\end{frame}

\section{Effects}

\subsection{Grace}

\begin{frame}
\begin{itemize}
\item  Gives sanctifying grace.
\item  Forgives sins.
\end{itemize}
\end{frame}

\subsection{Forgiveness}

\begin{frame}{Removes remnants of sin}
Removes
\begin{itemize}
\item weaknesses in the soul.
\item weaknesses in the body. (bad habits)
\item temporal punishment.
\end{itemize}
\end{frame}

\begin{frame}{Gives strength and courage}
It brings to the soul  a  supernatural  peace  and
resignation, and with these  great  strength  to  meet  death  bravely,  and
overcome all the attacks of the devil.
\end{frame}

\subsection{Healing}

\begin{frame}{Healing}
May restore bodily health
\begin{itemize}
\item If God's will.
\item  If given in time.
\item    If patient wills it and has faith in its power.
\end{itemize}
\end{frame}

\section{When}

\subsection{Danger}

\begin{frame}{Warning}
\begin{itemize}
 \item Not to delay unreasonably.
 \item Why.
\end{itemize}
\end{frame}

\subsection{Who}

\begin{frame}{Subject}
\begin{itemize}
 \item Baptised.
 \item Adult.
 \item Danger of death from some intrinsic cause.
\end{itemize}
\end{frame}

\subsection{Why}

\begin{frame}{Value}
\begin{itemize}
 \item For immediate entry to heaven.
 \item Need to make ourselves worthy to receive it.
 \end{itemize}
\end{frame}

   

\end{document}

