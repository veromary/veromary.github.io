% This file was converted to LaTeX by Writer2LaTeX ver. 1.4
% see http://writer2latex.sourceforge.net for more info
\documentclass[a5paper]{article}
%\usepackage[ascii]{inputenc}
%\usepackage[T1]{fontenc}
\usepackage{fontspec}
\usepackage{libertine}
\usepackage[english]{babel}
\usepackage{amsmath}
\usepackage{amssymb,amsfonts,textcomp}
\newfontfamily\jarman{JARDOTTY}
\usepackage{color}
\usepackage[top=2cm,bottom=2cm,left=2cm,right=2cm,nohead,nofoot]{geometry}
\usepackage{array}
\usepackage{hhline}
\usepackage{hyperref}
\hypersetup{colorlinks=true, linkcolor=blue, citecolor=blue, filecolor=blue, urlcolor=blue}
% Footnote rule
\setlength{\skip\footins}{0.119cm}
\renewcommand\footnoterule{\vspace*{-0.018cm}\setlength\leftskip{0pt}\setlength\rightskip{0pt plus 1fil}\noindent\textcolor{black}{\rule{0.25\columnwidth}{0.018cm}}\vspace*{0.101cm}}

\newcommand\textjarman[1]{{\jarman #1}}
\newcommand\answer[1]{\textbf{\textit{#1}}}
\newcounter{z}
\setcounter{z}{0}
\newcommand\spaces[1]{ \_\loop \ifnum\value{z} < #1
~\_%
\stepcounter{z}%
\repeat%
\setcounter{z}{0}}

\title{}
\begin{document}

\setlength{\parskip}{6pt plus2pt minus2pt}


\noindent Name:

\noindent {\Large Chapter XLVII -- The Mass As A Sacrifice}

1. Our primary duty is to worship God, since He is  our  \spaces{7},  \spaces{7}
and \spaces{7}. 

2. It (\textjarman{is}) (\textjarman{is not}) a greater crime against our nature  for  a
person to refuse to acknowledge his relationship to God than  it  is  for  a
child to refuse to acknowledge  his  relationship  to  his  parents.  

3.  In
worshipping God we are exercising the virtue of \spaces{7}. 

4. Our worship  is
shown in adoration especially, but this implies also \spaces{7}, \spaces{7}  and
\spaces{7}. 

5. The one act of worship that can be  offered  to  God  alone  is
called \spaces{7}. 

6. Sacrifice (\textjarman{is}) (\textjarman{is not}) natural to man. 

7.  In  it,  God
is honoured  precisely  inasmuch  as  He  is  our  \spaces{7},  \spaces{7},  and
\spaces{7}. 

8. In it we acknowledge God's supreme \spaces{7}  over  us,  and  we
express our  entire  \spaces{7}  to  Him.  

9.  These  inner  dispositions  are
expressed  externally  in  the  \spaces{7}  and  \spaces{7}  of  a  victim.  

10. Sacrifice is offered by a \spaces{7} who acts on behalf of the \spaces{7}  which
he represents. 

11. The central Sacrifice was that of \spaces{7}. 

12.  It  gave
infinite worship to the Father, because it was offered by His only  begotten
\spaces{7} who was both \spaces{7} and \spaces{7} of His own  Sacrifice.  

13.  The
sacrifices of the Old Law derived  their  efficacy  from  the  Sacrifice  of
\spaces{7}. 

14. Christ's acts and sufferings were of infinite  worth,  because
they were those of the \spaces{7}. 

15. By the double consecration at the  Last
Supper Christ freed Calvary from the limits of \spaces{7}  and  \spaces{7}.  

16. At the Last Supper Christ (\textjarman{did}) (\textjarman{did not}) offer the first Mass. 

17.  At  the
Supper He made sure that Mass would be offered everywhere to the end of  the
world by saying to His Apostles: “Do \spaces{7} in commemoration of  Me.”  

18. Christ in the Eucharist (\textjarman{is}) (\textjarman{is not}) distant from Himself  in  heaven.  

19. At the Mass Christ (\textjarman{does}) (\textjarman{does not}) come down from  heaven.  

20.  A  single
consecration – that of the bread, for instance  –  (\textjarman{would})  (\textjarman{would  not})  be
enough to constitute a Mass. 

21. Christ's Sacrifice on the Cross is  put  at
our disposal by means of the  \spaces{7}.  

22.  The  Mass  is  (\textjarman{substantially})
(\textjarman{accidentally}) the same sacrifice as that of the Cross.  

23.  The  efficient
cause in any sacrifice is the  \spaces{7}.  

24.  The  material  cause  in  any
sacrifice is the \spaces{7}. 

25. The formal cause  in  any  sacrifice  is  the
\spaces{7}. 

26. The final cause in any  sacrifice  is  the  \spaces{7}.  

27.  We
(\textjarman{should}) (\textjarman{should not dare}) offer ourselves with Christ in  every  Mass.  

28. We should strive to have in our hearts the same dispositions as  those  that
were in Christ's Heart at the \spaces{7} and on the  \spaces{7}.  

29.  At  every
Mass we should strive as far as possible to receive \spaces{7}. 

30. We  should
strive to live the Mass all day and every day  by  conforming  our  \spaces{7}
completely to God's.



\newpage



1. Our primary duty is to worship God, since He is  our  \answer{Creator},  \answer{Conserver}
and \answer{Last End}. 
2. It (\answer{is}) a greater crime against our nature  for  a
person to refuse to acknowledge his relationship to God than  it  is  for  a
child to refuse to acknowledge  his  relationship  to  his  parents.  
3.  In
worshipping God we are exercising the virtue of \answer{Justice}. 
4. Our worship  is
shown in adoration especially, but this implies also \answer{thanksgiving}, \answer{reparation}  and
\answer{petition}. 
5. The one act of worship that can be  offered  to  God  alone  is
called \answer{sacrifice}. 
6. Sacrifice (\answer{is}) natural to man. 
7.  In  it,  God
is honoured  precisely  inasmuch  as  He  is  our  \answer{Creator},  \answer{Conserver},  and
\answer{Last End}. 
8. In it we acknowledge God's supreme \answer{dominion}  over  us,  and  we
express our  entire  \answer{subjection}  to  Him.  
9.  These  inner  dispositions  are
expressed  externally  in  the  \answer{oblation}  and  \answer{immolation}  of  a  victim.  
10.
Sacrifice is offered by a \answer{priest} who acts on behalf of the \answer{community}  which
he represents. 
11. The central Sacrifice was that of \answer{Calvary}. 
12.  It  gave
infinite worship to the Father, because it was offered by His only  begotten
\answer{Son} who was both \answer{Priest} and \answer{Victim} of His own  Sacrifice.  
13.  The
sacrifices of the Old Law derived  their  efficacy  from  the  Sacrifice  of
\answer{Calvary}. 
14. Christ's acts and sufferings were of infinite  worth,  because
they were those of the \answer{God}. 
15. By the double consecration at the  Last
Supper Christ freed Calvary from the limits of \answer{time}  and  \answer{place}.  
16.
At the Last Supper Christ (\answer{did}) offer the first Mass. 
17.  At  the
Supper He made sure that Mass would be offered everywhere to the end of  the
world by saying to His Apostles: “Do \answer{this} in commemoration of  Me.”  
18.
Christ in the Eucharist (\answer{is not}) distant from Himself  in  heaven.  
19.
At the Mass Christ (\answer{does not}) come down from  heaven.  
20.  A  single
consecration --– that of the bread, for instance  ---  (\answer{would  not})  be
enough to constitute a Mass. 
21. Christ's Sacrifice on the Cross is  put  at
our disposal by means of the  \answer{Mass}.  
22.  The  Mass  is  (\answer{substantially})
(accidentally) the same sacrifice as that of the Cross.  
23.  The  efficient
cause in any sacrifice is the  \answer{priest}.  
24.  The  material  cause  in  any
sacrifice is the \answer{victim}. 
25. The formal cause  in  any  sacrifice  is  the
\answer{immolation}. 
26. The final cause in any  sacrifice  is  the  \answer{worship due to God}.  
27.  We
(\answer{should}) (should not dare) offer ourselves with Christ in  every  Mass.  
28. We should strive to have in our hearts the same dispositions as  those  that
were in Christ's Heart at the \answer{Last Supper} and on the  \answer{Cross}.  
29.  At  every
Mass we should strive as far as possible to receive \answer{Communion}. 
30. We  should
strive to live the Mass all day and every day  by  conforming  our  \answer{will}
completely to God's.



\end{document}
