% This file was converted to LaTeX by Writer2LaTeX ver. 1.4
% see http://writer2latex.sourceforge.net for more info
\documentclass[a4paper,14pt]{memoir}
%\usepackage[ascii]{inputenc}
%\usepackage[T1]{fontenc}
\usepackage{fontspec}
\usepackage{libertine}
\usepackage[english]{babel}
\usepackage{amsmath}
\usepackage{amssymb,amsfonts,textcomp}
\newfontfamily\jarman{JARDOTTY}
\usepackage{color}
\usepackage[top=2cm,bottom=2cm,left=2cm,right=2cm,nohead,nofoot]{geometry}
\usepackage{array}
\usepackage{hhline}
\usepackage{hyperref}
\hypersetup{colorlinks=true, linkcolor=blue, citecolor=blue, filecolor=blue, urlcolor=blue}
% Footnote rule
\setlength{\skip\footins}{0.119cm}
\renewcommand\footnoterule{\vspace*{-0.018cm}\setlength\leftskip{0pt}\setlength\rightskip{0pt plus 1fil}\noindent\textcolor{black}{\rule{0.25\columnwidth}{0.018cm}}\vspace*{0.101cm}}

\newcommand\textjarman[1]{{\jarman #1}}
\newcommand\answer[1]{\textbf{\textit{#1}}}
\newcounter{z}
\setcounter{z}{0}
\newcommand\spaces[1]{ \_\loop \ifnum\value{z} < #1
~\_%
\stepcounter{z}%
\repeat%
\setcounter{z}{0}}

\title{}
\begin{document}

\setlength{\parskip}{6pt plus2pt minus2pt}


\noindent Name:

\noindent {\Large Chapter XLVIII -- The Liturgy of the Mass}




1. The  chasuble  was  originally  a  kind  of  \spaces{7};  the  maniple  was
\spaces{7}; the stole was \spaces{7};  the  Alb  was  \spaces{7};  the  amice  was
\spaces{7}. 

2. The vestments are in five colours, namely: \spaces{7}, \spaces{7}, \spaces{7}, \spaces{7}, and \spaces{7}. 

3. On  top
of the chalice you find first the \spaces{7}; then  the  \spaces{7}  ;  then  the
\spaces{7}; then the \spaces{7} which contains the \spaces{7}. 

4. The altar  reminds
us of a \spaces{7} and of a \spaces{7}. 

5. Imbedded in it is the \spaces{7}  which
contains \spaces{7}. 

6. The safe-like structure  in  which  the  Eucharist  is
kept is called the \spaces{7}. 

7. The sacred  vessel  in  which  the  Hosts  are
reserved is called a  \spaces{7}.  

8.  The  altar  is  covered  with  \spaces{7}
cloths. 

9. The candles remind  us  of  \spaces{7}.  

10.  The  Cross  over  the
tabernacle reminds us of \spaces{7}. 

11. The Mass is usually divided into  two
main parts; that of the \spaces{7} and that of the \spaces{7}. 

12. We  begin  Mass
with the (\textjarman{small}) (\textjarman{large}) Sign of the Cross. 

13. In  psalm  42  we  beg  that
\spaces{7}. 

14. When saying the Confiteor, we should picture to ourselves  the
\spaces{7}, and make a very fervent act of \spaces{7}.  

15.  The  priest  kisses
the altar because \spaces{7} and because \spaces{7}. 

16. The \spaces{7} gives  the
leading thought in the Mass to be offered. 

17.  Kyrie  eleison  and  Christe
eleison mean  \spaces{4}  \spaces{4} \spaces{4} and  \spaces{4}  \spaces{4}  \spaces{4} respectively. 

18. The Gloria begins  with
the song of the \spaces{7}. 

19.  It  gives  praise  and  glory  first  to  the
\spaces{7}; then to the \spaces{7}; and lastly, to the \spaces{4} \spaces{7}.  

20.  It  is  a
perfect expression  of  the  dispositions  we  should  have  in  offering  a
sacrifice, namely, those of \spaces{7}, \spaces{7}, \spaces{7}  and  \spaces{7}  

21.
In the Collect, there is first the \spaces{7} to God;  then  the  \spaces{7}  of
the petition; and lastly, the \spaces{7} itself. 

22. As far as the Collect  in
the Mass we speak to God; then He speaks to  us  in  the  \spaces{7}  and  the
\spaces{7}. 

23. The Gradual and the Tract look to the (\textjarman{Epistle}) (\textjarman{Gospel});  the
Alleluia looks to the (\textjarman{Epistle}) (\textjarman{Gospel}). 

24. There are  \spaces{7}  different
Sequences today. 

25. The first part of the Mass ends with the \spaces{7}.  

26.
The main part of the Mass begins at the \spaces{7}. 

27. The union of  the  few
drops of water with the wine put into the chalice represents the union  that
should exist between us and \spaces{7} by  \spaces{7}.  

28.  After  washing  his
fingers the priest says: “Pray brethren, that MY sacrifice and \spaces{7}  may
be acceptable to God the Father Almighty.” 

29. This  shows  that  we  should
offer \spaces{7} as well as Our Lord in the Mass. 

30. The Offertory ends  with
the \spaces{7} prayer. 

31. There are \spaces{7} different Prefaces. 

32. Each  is
divided into three parts: in the first,  we  \spaces{7} \spaces{7};  in  the  second,  we
\spaces{7} \spaces{7}; in the third, we \spaces{7} \spaces{7}. 

33. The bell is rung at the sanctus  to
remind us that the \spaces{7} of the Mass is about to begin. 

34.  The  “canon”
has remained for over \spaces{7} years. 

35. At the consecration the  Body  and
Blood of Christ are separated \spaces{7}. 

36. His \spaces{7} is thus freed  from
the limits of time and place and made present before us  on  the  altar;  so
that we can offer to the Father is our sacrifice. 

37. The “canon” ends  with
the doxology: \spaces{7}  \spaces{7} \spaces{7} \spaces{7} \spaces{7} \spaces{7}.  

38.  The  Our  Father  is  said  as  an  immediate
preparation for \spaces{7}. 

39. As the priest  gives  us  Holy  Communion,  he
says: “May the \spaces{7} preserve thy soul unto \spaces{7}.”  

40.  In  all  the
prayers of the ordinary of the Mass, the word “love”  does  not  occur  even
once.  This  is  because  the  Mass  is  a  \spaces{7};  hence  the   dominant
disposition is not that  of  charity  but  that  of  \spaces{7}  by  which  we
acknowledge  God's  supreme  excellence,  His  dominion  over  us,  and  our
absolute subjection to Him as our \spaces{7} , \spaces{7} and \spaces{7}.


\newpage





1. The  chasuble  was  originally  a  kind  of  \answer{cloak};  the  maniple  was
\answer{a small towel}; the stole was \answer{a handkerchief};  the  Alb  was  \answer{an inner garment};  the  amice  was
\answer{a scarf}. 
2. The vestments are in five colours, namely: \answer{white, red, green, violet and black}. 
3. On  top
of the chalice you find first the \answer{paten}; then  the  \answer{veil}  ;  then  the
\answer{pall}; then the \answer{burse} which contains the \answer{corporal}. 
4. The altar  reminds
us of a \answer{table} and of a \answer{tomb}. 
5. Imbedded in it is the \answer{stone}  which
contains \answer{relics}. 
6. The safe-like structure  in  which  the  Eucharist  is
kept is called the \answer{tabernacle}. 
7. The sacred  vessel  in  which  the  Hosts  are
reserved is called a  \answer{ciborium}.  
8.  The  altar  is  covered  with  \answer{three}
cloths. 
9. The candles remind  us  of  \answer{the darkness of the catacombs}.  
10.  The  Cross  over  the
tabernacle reminds us of \answer{the death of Christ}. 
11. The Mass is usually divided into  two
main parts; that of the \answer{Catechumens} and that of the \answer{Faithful}. 
12. We  begin  Mass
with the (small) (\answer{large}) Sign of the Cross. 
13. In  psalm  42  we  beg  that
\answer{Judge me and defend me from my enemies}. 
14. When saying the Confiteor, we should picture to ourselves  the
\answer{Last Judgement}, and make a very fervent act of \answer{contrition}.  
15.  The  priest  kisses
the altar because \answer{it represents Our Lord} and because \answer{it contains relics}. 
16. The \answer{Introit} gives  the
leading thought in the Mass to be offered. 
17.  Kyrie  eleison  and  Christe
eleison mean \answer{Lord have mercy} and \answer{Christ have mercy} respectively. 
18. The Gloria begins  with
the song of the \answer{angels}. 
19.  It  gives  praise  and  glory  first  to  the
\answer{Father}; then to the \answer{Son}; and lastly, to the \answer{Holy Ghost}.  
20.  It  is  a
perfect expression  of  the  dispositions  we  should  have  in  offering  a
sacrifice, namely, those of \answer{Adoration}, \answer{Thanksgiving}, \answer{Contrition}  and  \answer{Petition}  
21.
In the Collect, there is first the \answer{Address} to God;  then  the  \answer{Motive}  of
the petition; and lastly, the \answer{Petition} itself. 
22. As far as the Collect  in
the Mass we speak to God; then He speaks to  us  in  the  \answer{Epistle}  and  the
\answer{Gospel}. 
23. The Gradual and the Tract look to the (\answer{Epistle}) (Gospel);  the
Alleluia looks to the (Epistle) (\answer{Gospel}). 
24. There are  \answer{five}  different
Sequences today. 
25. The first part of the Mass ends with the \answer{Creed}.  
26.
The main part of the Mass begins at the \answer{Offertory}. 
27. The union of  the  few
drops of water with the wine put into the chalice represents the union  that
should exist between us and \answer{Christ} by  \answer{grace}.  
28.  After  washing  his
fingers the priest says: “Pray brethren, that MY sacrifice and \answer{yours}  may
be acceptable to God the Father Almighty.” 
29. This  shows  that  we  should
offer \answer{ourselves} as well as Our Lord in the Mass. 
30. The Offertory ends  with
the \answer{Secret} prayer. 
31. There are \answer{fifteen} different Prefaces. 
32. Each  is
divided into three parts: in the first,  we  \answer{praise and give thanks};  in  the  second,  we
\answer{mention a particular motive}; in the third, we \answer{unite our prayer with the angels}. 
33. The bell is rung at the sanctus  to
remind us that the \answer{canon} of the Mass is about to begin. 
34.  The  “canon”
has remained for over \answer{two thousand} years. 
35. At the consecration the  Body  and
Blood of Christ are separated \answer{sacramentally}. 
36. His \answer{death} is thus freed  from
the limits of time and place and made present before us  on  the  altar;  so
that we can offer to the Father is our sacrifice. 
37. The “canon” ends  with
the doxology:  \answer{through Him and with Him and in Him}.  
38.  The  Our  Father  is  said  as  an  immediate
preparation for \answer{Holy Communion}. 
39. As the priest  gives  us  Holy  Communion,  he
says: “May the \answer{Body of Christ} preserve thy soul unto \answer{everlasting life}.”  
40.  In  all  the
prayers of the ordinary of the Mass, the word “love”  does  not  occur  even
once.  This  is  because  the  Mass  is  a  \answer{sacrifice};  hence  the   dominant
disposition is not that  of  charity  but  that  of  \answer{justice?}  by  which  we
acknowledge  God's  supreme  excellence,  His  dominion  over  us,  and  our
absolute subjection to Him as our \answer{Creator}, \answer{Conserver} and \answer{Last End}.

\end{document}
