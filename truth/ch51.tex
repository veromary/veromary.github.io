% This file was converted to LaTeX by Writer2LaTeX ver. 1.4
% see http://writer2latex.sourceforge.net for more info
\documentclass[a4paper,14pt]{memoir}
%\usepackage[ascii]{inputenc}
%\usepackage[T1]{fontenc}
\usepackage{fontspec}
\usepackage{libertine}
\usepackage[english]{babel}
\usepackage{amsmath}
\usepackage{amssymb,amsfonts,textcomp}
\newfontfamily\jarman{JARDOTTY}
\usepackage{color}
\usepackage[top=2cm,bottom=2cm,left=2cm,right=2cm,nohead,nofoot]{geometry}
\usepackage{array}
\usepackage{hhline}
\usepackage{hyperref}
\hypersetup{colorlinks=true, linkcolor=blue, citecolor=blue, filecolor=blue, urlcolor=blue}
% Footnote rule
\setlength{\skip\footins}{0.119cm}
\renewcommand\footnoterule{\vspace*{-0.018cm}\setlength\leftskip{0pt}\setlength\rightskip{0pt plus 1fil}\noindent\textcolor{black}{\rule{0.25\columnwidth}{0.018cm}}\vspace*{0.101cm}}

\newcommand\textjarman[1]{{\jarman #1}}
\newcommand\answer[1]{\textbf{\textit{#1}}}
\newcounter{z}
\setcounter{z}{0}
\newcommand\spaces[1]{\rule{0pt}{18pt} \_\loop \ifnum\value{z} < #1
~\_%
\stepcounter{z}%
\repeat%
\setcounter{z}{0}}

\title{}
\begin{document}

\setlength{\parskip}{6pt plus2pt minus2pt}


\noindent Name:

\noindent {\Large Chapter LI -- Matrimony}

1. Like Orders, Matrimony is  a  \spaces{7}  sacrament.  

2.  Sex  has  two  main
functions: it is an expression of \spaces{7}; and it enables husband and  wife
to co-operate with God in the creation of \spaces{7}. 

3. Marriage is a  \spaces{7}
since it involves the giving and accepting of a right. 

4. It  is  a  special
contract, because 
\begin{itemize}
\item it \spaces{7},
\item it \spaces{7},
\item and it  \spaces{7}.  
\end{itemize}

5.  Its
efficient  cause  is  \spaces{7} \spaces{7};  its  formal  cause  is  \spaces{7};  its
material cause is \spaces{7}; and its final  cause  is  \spaces{7} \spaces{7} \spaces{7} \spaces{7}.  

6.  Its
primary end is \spaces{7} \spaces{7}; its secondary end is \spaces{7}.  

7.  The  matter
and form in the  sacrament  is  found  in  \spaces{7}  

8.  The  ministers  are
\spaces{7}. 

9. As a sacrament, marriage comes  under  the  authority  of  the
(\textjarman{Church}) (\textjarman{State}). 

10. The two properties of marriage are its \spaces{7}  and
\spaces{12}.  

11.  Its  unity  forbids  \spaces{7}  and  \spaces{7}.   

12.   Its
indissolubility forbids  \spaces{7}.  

13.  Divorce  given  on  merely  human
authority is against the natural law, because it is opposed to the  good  of
\spaces{7},  \spaces{7} \spaces{7}  and  \spaces{7}.  

14.  The  only  exceptions   to   the
indissolubility of marriage are those granted by \spaces{7}  \spaces{7}  and  taught  by
\spaces{7}. 

15. Marriage between two baptised, when  consummated  (\textjarman{may  still})
(\textjarman{may never}) be dissolved. 

16. If not consummated, such  a  marriage  may  be
dissolved either by \spaces{7} \spaces{7}  or by \spaces{7} \spaces{7}   \spaces{7}.  

17.  Two  unbaptised  marry.
Later, one is baptised. This marriage may be dissolved either by \spaces{7}  \spaces{7}
or by \spaces{7} \spaces{7}.  

18. To marry validly, a  male  must  be  at  least  \spaces{7}
years old; a female, at least \spaces{7}.  

19.  Diversity  of  religion  is  a
(\textjarman{diriment}) (\textjarman{prohibitive}) impediment. 

20.  Mixed  marriage  is  a  (\textjarman{diriment})
(\textjarman{prohibitive}) impediment. 

21. Relationship  by  blood  is  called  \spaces{7};
relationship by marriage is called \spaces{7}. 

22. Four  conditions  required
that marriage be licit are \spaces{7}, \spaces{7},  \spaces{7}  and  \spaces{7}.


23. Five reasons why the Church forbids  Mixed  Marriages  are  as  follows:
\begin{enumerate}
\item \spaces{7} 
\item \spaces{7} 
\item \spaces{7} 
\item \spaces{7}  
\item \spaces{7}.  
\end{enumerate}


24. Catholics (\textjarman{are})  (\textjarman{are
not}) allowed to go with Non-Catholics with a view to marriage.  

25.  Notices
of intended marriages are called \spaces{7}. 

26. Children belong primarily  to
the (\textjarman{parents}) (\textjarman{state}). 

27. Parents (\textjarman{are}) (\textjarman{are not}) obliged to educate  their
children  in  Catholic  schools.  

28.  A  school  (\textjarman{can})  (\textjarman{cannot})  make   up
completely  for  lack  of  home  training.  

29.  Children  are  obliged   to
\spaces{7}, \spaces{7} and \spaces{7} their parents.

\newpage

1. Like Orders, Matrimony is  a  \answer{social}  sacrament. 
2.  Sex  has  two  main
functions: it is an expression of \answer{love}; and it enables husband and  wife
to co-operate with God in the creation of \answer{a soul}.
3. Marriage is a  \answer{contract}
since it involves the giving and accepting of a right.
4. It  is  a  special
contract, because it \answer{concerns persons}, it \answer{was instituted by God}, and it  \answer{is a sacrament}. 
5.  Its
efficient  cause  is  \answer{the inner consent};  its  formal  cause  is  \answer{the bond};  its
material cause is \answer{the two persons}; and its final  cause  is  \answer{the generation and education of children and home life}. 
6.  Its
primary end is \answer{the generation and education of children}; its secondary end is \answer{home life}. 
7.  The  matter
and form in the  sacrament  is  found  in  \answer{the consent} 
8.  The  ministers  are
\answer{the two persons}.
9. As a sacrament, marriage comes  under  the  authority  of  the
(\answer{Church}) (State).
10. The two properties of marriage are its \answer{unity}  and
\answer{indissolubility}. 
11.  Its  unity  forbids  \answer{polygamy}  and  \answer{polyandry}.  
12.   Its
indissolubility forbids  \answer{divorce}. 
13.  Divorce  given  on  merely  human
authority is against the natural law, because it is opposed to the  good  of
\answer{the children},  \answer{the husband and wife}  and  \answer{the state}. 
14.  The  only  exceptions   to   the
indissolubility of marriage are those granted by  \answer{Papal dispensation}  and  taught  by
\answer{St Paul}.
15. Marriage between two baptised, when  consummated  (may  still)
(\answer{may never}) be dissolved.
16. If not consummated, such  a  marriage  may  be
dissolved either by \answer{Papal dispensation} or by \answer{solemn religious profession}. 
17.  Two  unbaptised  marry.
Later, one is baptised. This marriage may be dissolved either by  \answer{Papal dispensation}
or by \answer{Pauline privilege}.
18. To marry validly, a  male  must  be  at  least  \answer{16}
years old; a female, at least \answer{14}. 
19.  Diversity  of  religion  is  a
(\answer{diriment}) (prohibitive) impediment.
20.  Mixed  marriage  is  a  (diriment)
(\answer{prohibitive}) impediment.
21. Relationship  by  blood  is  called  \answer{consanguinity};
relationship by marriage is called \answer{affinity}.
22. Four  conditions  required
that marriage be licit are \answer{state of grace}, \answer{free from impediments},  \answer{sufficiently instructed}  and  \answer{observe the laws}.
23. Five reasons why the Church forbids  Mixed  Marriages  are  as  follows:
\answer{division}, \answer{danger of divorce}, \answer{danger of perversion of children}, \answer{impossible for children to be rightly educated}, \answer{dissension concerning vocations and moral matters}.
24. Catholics (are)  (\answer{are
not}) allowed to go with Non-Catholics with a view to marriage. 
25.  Notices
of intended marriages are called \answer{Wedding Bans}.
26. Children belong primarily  to
the (\answer{parents}) (state).
27. Parents (\answer{are}) (are not) obliged to educate  their
children  in  Catholic  schools. 
28.  A  school  (can)  (\answer{cannot})  make   up
completely  for  lack  of  home  training. 
29.  Children  are  obliged   to
\answer{love}, \answer{respect} and \answer{obey} their parents.

\end{document}

