% This file was converted to LaTeX by Writer2LaTeX ver. 1.4
% see http://writer2latex.sourceforge.net for more info
\documentclass[a5paper]{article}
\usepackage[ascii]{inputenc}
\usepackage[T1]{fontenc}
\usepackage{fontspec}
\usepackage{libertine}
\usepackage[english]{babel}
\usepackage{amsmath}
\usepackage{amssymb,amsfonts,textcomp}
\newfontfamily\jarman{JARDOTTY}
\usepackage{color}
\usepackage[top=2cm,bottom=2cm,left=2cm,right=2cm,nohead,nofoot]{geometry}
\usepackage{array}
\usepackage{hhline}
\usepackage{hyperref}
\hypersetup{colorlinks=true, linkcolor=blue, citecolor=blue, filecolor=blue, urlcolor=blue}
% Footnote rule
\setlength{\skip\footins}{0.119cm}
\renewcommand\footnoterule{\vspace*{-0.018cm}\setlength\leftskip{0pt}\setlength\rightskip{0pt plus 1fil}\noindent\textcolor{black}{\rule{0.25\columnwidth}{0.018cm}}\vspace*{0.101cm}}

\newcommand\textjarman[1]{{\jarman #1}}
\newcommand\answer[1]{\textbf{\textit{#1}}}
\newcounter{z}
\setcounter{z}{0}
\newcommand\spaces[1]{ \_\loop \ifnum\value{z} < #1
~\_%
\stepcounter{z}%
\repeat%
\setcounter{z}{0}}

\title{}
\begin{document}

\setlength{\parskip}{6pt plus2pt minus2pt}


\noindent Name:

\noindent {\Large Chapter XLIV -- Penance}

1. Christ worked a special \spaces{7} to prove He could forgive sins.

2. He gave this power to His Apostles when He said to them: “Whose sins \spaces{7}.”

3. This power (\textjarman{is}) (\textjarman{is not}) universal; it is to be exercised after the manner of a \spaces{7}; hence, it implies the \spaces{7} of our sins; and this judgment is ratified by \spaces{7}.

4. Penance is a distinct sacrament from baptism since it differs from it in these five ways: 
\begin{enumerate}
\item \spaces{7}
\item \spaces{7}
\item \spaces{7}
\item \spaces{7}
\item \spaces{7}
\end{enumerate}

5. In penance the Sign Only is \spaces{7}; the Thing and Sign is \spaces{7}; and Thing Only is \spaces{7}.

6. The remote matter is all \spaces{7} sins committed after \spaces{7}; the proximate matter is \spaces{7}.

7. The motive is a (\textjarman{natural}) (\textjarman{supernatural}) one.

8. It (\textjarman{is}) (\textjarman{is not}) possible to have true sorrow without having with it the resolution to avoid sin in future.

9. Contrition is in our (\textjarman{feelings}) (\textjarman{will}) (\textjarman{intellect}) (\textjarman{imagination}).

10. What makes contrition perfect, as opposed to imperfect, is its (\textjarman{sincerity}) (\textjarman{intensity}) (\textjarman{motive}).

11. In perfect contrition the motive flows from supernatural \spaces{7}, which is a true friendship between the soul and \spaces{7}.

12. If I am sorry for sin because Our Lord has been so good to me, my contrition is (\textjarman{perfect}) (\textjarman{imperfect}).

13. If I am sorry for sin because I fear hell, my contrition is \spaces{7}.

14. If I am sorry for sin because God is infinitely good and lovable in Himself, my contrition is \spaces{7}.

15. (\textjarman{Perfect}) (\textjarman{Imperfect}) contrition forgives sin even apart from the actual reception of a sacrament.

16. Contrition should have these four qualities: it should be \spaces{7}, \spaces{7}, \spaces{7} and \spaces{7}.

17. If I am firmly determined to take the \spaces{7} to avoid sin in future, my contrition is sure to be supreme.

18. If a person's contrition is true but not supreme, his confession is (\textjarman{invalid}) (\textjarman{valid but not fruitful}) (\textjarman{valid and fruitful}).

19. We should prepare for confession by praying for grace and by \spaces{7}.

20. We should spend most of the time (\textjarman{examining our conscience}) (\textjarman{trying to gain and increase our contrition}).

21. The more intense my act of contrition is, the more \spaces{7} I will get; the more \spaces{7} it will abolish, and the more it will guarantee me against \spaces{7}.

22. I am (\textjarman{free}) (\textjarman{obliged}) to tell formal mortal sins committed after baptism and not yet mentioned in a good confession.

23. I must tell them according to their \spaces{7} and their \spaces{7}.

24. I (\textjarman{am}) (\textjarman{am not}) obliged to say my penance before my next confession.

25. If I forget to tell a mortal sin in confession I should (\textjarman{go straight back and tell it}) (\textjarman{tell it the next time I normally go to confession}).

26. If I forget what my penance was I should (\textjarman{ask another penitent what his was}) (\textjarman{make up one for myself}) (\textjarman{say the one I got last time}) (\textjarman{mention it at the next confession and ask for one to take its place}).

27. Confession benefits the individual in these five ways: it \spaces{7}, \spaces{7}, \spaces{7}, \spaces{7}, \spaces{7},

28. It benefits society also by promoting the practice of the two great social virtues of \spaces{7} and \spaces{7}.

\newpage

1. Christ worked a special \answer{miracle} to prove He could forgive sins.
2. He gave this power to His Apostles when He said to them: “Whose sins \answer{you shall forgive, they are forgiven them}.”
3. This power (\answer{is}) (is not) universal; it is to be exercised after the manner of a \answer{judgement}; hence, it implies the \answer{confession} of our sins; and this judgment is ratified by \answer{God}.
4. Penance is a distinct sacrament from baptism since it differs from it in these five ways: \answer{1. Baptism for outsiders, penance is for members}, \answer{2. Baptism is a new birth, penance is resurrection}, \answer{3. Penance like a judgement, not baptism}, \answer{4. Baptism gives a character, penance doesn't}, \answer{5. Baptism can be given by a layperson, penance only by a priest}.
5. In penance the Sign Only is \answer{external confession}; the Thing and Sign is \answer{spiritual reality put into the soul}; and Thing Only is \answer{the grace given}.
6. The remote matter is all \answer{formal} sins committed after \answer{baptism}; the proximate matter is \answer{three acts: Contrition, Confession \& Satisfaction}.
7. The motive is a (natural) (\answer{supernatural}) one.
8. It (is) (\answer{is not}) possible to have true sorrow without having with it the resolution to avoid sin in future.
9. Contrition is in our (feelings) (\answer{will}) (intellect) (imagination).
10. What makes contrition perfect, as opposed to imperfect, is its (sincerity) (intensity) (\answer{motive}).
11. In perfect contrition the motive flows from supernatural \answer{charity}, which is a true friendship between the soul and \answer{God}.
12. If I am sorry for sin because Our Lord has been so good to me, my contrition is (perfect) (\answer{imperfect}).
13. If I am sorry for sin because I fear hell, my contrition is \answer{imperfect}.
14. If I am sorry for sin because God is infinitely good and lovable in Himself, my contrition is \answer{perfect}.
15. (\answer{Perfect}) (Imperfect) contrition forgives sin even apart from the actual reception of a sacrament.
16. Contrition should have these four qualities: it should be \answer{internal}, \answer{supernatural}, \answer{universal} and \answer{supreme}.
17. If I am firmly determined to take the \answer{path/steps/means} to avoid sin in future, my contrition is sure to be supreme.
18. If a person's contrition is true but not supreme, his confession is (invalid) (\answer{valid but not fruitful}) (valid and fruitful).
19. We should prepare for confession by praying for grace and by \answer{meditating/examining conscience}.
20. We should spend most of the time (examining our conscience) (\answer{trying to gain and increase our contrition}).
21. The more intense my act of contrition is, the more \answer{grace} I will get; the more \answer{temporal punishment} it will abolish, and the more it will guarantee me against \answer{future falls}.
22. I am (free) (\answer{obliged}) to tell formal mortal sins committed after baptism and not yet mentioned in a good confession.
23. I must tell them according to their \answer{number} and their \answer{kind}.
24. I (am) (\answer{am not}) obliged to say my penance before my next confession.
25. If I forget to tell a mortal sin in confession I should (go straight back and tell it) (\answer{tell it the next time I normally go to confession}).
26. If I forget what my penance was I should (ask another penitent what his was) (make up one for myself) (say the one I got last time) (\answer{mention it at the next confession and ask for one to take its place}).
27. Confession benefits the individual in these five ways: it \answer{gives grace}, \answer{helps avoid sin}, \answer{banishes temporal pumishment}, \answer{peace of mind}, \answer{excellent advice},
28. It benefits society also by promoting the practice of the two great social virtues of \answer{justice} and \answer{charity}.

\end{document}
