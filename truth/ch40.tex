% This file was converted to LaTeX by Writer2LaTeX ver. 1.4
% see http://writer2latex.sourceforge.net for more info
\documentclass[a4paper]{article}
\usepackage[ascii]{inputenc}
\usepackage[T1]{fontenc}
\usepackage{fontspec}
\usepackage{libertine}
\newfontfamily\jarman{JARDOTTY}
\usepackage[english]{babel}
\usepackage{amsmath}
\usepackage{amssymb,amsfonts,textcomp}
\usepackage{color}
\usepackage[top=2cm,bottom=2cm,left=2cm,right=2cm,nohead,nofoot]{geometry}
\usepackage{array}
\usepackage{hhline}
\usepackage{hyperref}
\hypersetup{colorlinks=true, linkcolor=blue, citecolor=blue, filecolor=blue, urlcolor=blue}
% Footnote rule
\setlength{\skip\footins}{0.119cm}
\renewcommand\footnoterule{\vspace*{-0.018cm}\setlength\leftskip{0pt}\setlength\rightskip{0pt plus 1fil}\noindent\textcolor{black}{\rule{0.25\columnwidth}{0.018cm}}\vspace*{0.101cm}}
\title{}

\newcommand\textjarman[1]{{\jarman #1}}
\newcommand\answer[1]{\textbf{\textit{#1}}}
\newcounter{z}
\setcounter{z}{0}
\newcommand\spaces[1]{ \_\loop \ifnum\value{z} < #1
~\_%
\stepcounter{z}%
\repeat%
\setcounter{z}{0}}

\begin{document}
Name:

Chapter XXXX -- Gifts of the Holy Ghost

1. The Gifts of the Holy Ghost are (permanent) (passing) helps.

2. There are \ \_ \_ \_ \_ \_ \_ \_ \ of them, and their names are

\begin{itemize}
\item \_ \_ \_ \_ \_ \_ \_ \ \ 
\item \_ \_ \_ \_ \_ \_ \_ \ \ 
\item \_ \_ \_ \_ \_ \_ \_ \ \ 
\item \_ \_ \_ \_ \_ \_ \_ \ \ 
\item \_ \_ \_ \_ \_ \_ \_ \ \ 
\item \_ \_ \_ \_ \_ \_ \_ \ 
\item and \ \_ \_ \_ \_ \_ \_ \_ .
\end{itemize}
3. They are found in our (soul) (spiritual faculties).

4. They come with \ \_ \_ \_ \_ \_ \_ \_ .

5. They are increased by the Sacrament of \ \_ \_ \_ \_ \_ \_ \_ ; and by any increase in the intensity of our \ \_ \_ \_ \_ \_ \_ \_ .

6. They are lost by \ \_ \_ \_ \_ \_ \_ \_ .

7. They function in a (human) (divine) way.

8. They make us exquisitely \ \_ \_ \_ \_ \_ \_ \_ \ and \ \_ \_ \_ \_ \_ \_ \_ \ to the \ \_ \_ \_ \_ \_ \_ \_ \ and the \ \_ \_ \_ \_ \_ \_ \_ \ of the Holy Ghost.

9. They are an absolutely safe road to Christian \ \_ \_ \_ \_ \_ \_ \_ , and the \ \_ \_ \_ \_ \_ \_ \_ \ sanctity of the saints.

10. The beatitudes are so called because they show us how to be truly \ \_ \_ \_ \_ \_ \_ \_ \ in this life and in the next.

11. St. Paul mentions \ \_ \_ \_ \_ \_ \_ \_ \ fruits of the Holy Ghost; and he contrasts them to the fruits of the \ \_ \_ \_ \_ \_ \_ \_ .

12. The beatitudes are excellent (actions) (habits).

13. The least excellent of the Gifts is that called \ \_ \_ \_ \_ \_ \_ \_ ; the most excellent, that called \ \_ \_ \_ \_ \_ \_ \_ .

14. Prudence is complemented by the Gift of \ \_ \_ \_ \_ \_ \_ \_ .

15. It is the Gift of \ \_ \_ \_ \_ \_ \_ \_ \ which makes us judge rightly of creatures.

16. Hence this Gift helps the virtue of \ \_ \_ \_ \_ \_ \_ \_ .

17. It is the Gift of \ \_ \_ \_ \_ \_ \_ \_ \ which gives us supernatural insight into God's revealed truths.

18. Therefore it assists the virtue of \ \_ \_ \_ \_ \_ \_ \_ .

19. Which are the Gifts which give infused contemplation? These are \ \_ \_ \_ \_ \_ \_ \_ , \ \_ \_ \_ \_ \_ \_ \_ \ and \ \_ \_ \_ \_ \_ \_ \_ .

20. Which give the highest form of contemplation? \ \_ \_ \_ \_ \_ \_ \_ .

21. Which of the virtues does the Gift of Piety complement? \ \_ \_ \_ \_ \_ \_ \_ .

22. To be truly happy, we must separate ourselves from what is sinful in \ \_ \_ \_ \_ \_ \_ \_ \ and unite ourselves to \ \_ \_ \_ \_ \_ \_ \_ .

23. We can be loyal to Christ only at the expense of \ \_ \_ \_ \_ \_ \_ \_ .

24. St. Paul speaks of us as ``Having nothing, and possessing \ \_ \_ \_ \_ \_ \_ \_ .''

25. When we have \ \_ \_ \_ \_ \_ \_ \_ , we have all.


\bigskip


\bigskip

\clearpage
1. The Gifts of the Holy Ghost are (\textbf{\textit{permanent}}) (passing) helps. 2. There are (\textbf{\textit{seven}}) of them, and their names are (\textbf{\textit{fear}}) (\textbf{\textit{piety}}) (\textbf{\textit{fortitude}}) (\textbf{\textit{counsel}}) (\textbf{\textit{knowledge}}) (\textbf{\textit{understanding}}) and (\textbf{\textit{wisdom}}). 3. They are found in our (soul) (\textbf{\textit{spiritual faculties}}). 4. They come with (\textbf{\textit{sanctifying grace}}). 5. They are increased by the Sacrament of (\textbf{\textit{Confirmation}}); and by any increase in the intensity of our (\textbf{\textit{charity}}). 6. They are lost by (\textbf{\textit{formal mortal sin}}). 7. They function in a (human) (\textbf{\textit{divine}}) way. 8. They make us exquisitely (\textbf{\textit{sensitive}}) and (\textbf{\textit{docile}}) to the (\textbf{\textit{illuminations}}) and the (\textbf{\textit{inspirations}}) of the Holy Ghost. 9. They are an absolutely safe road to Christian (\textbf{\textit{perfection}}), and the (\textbf{\textit{heroic}}) sanctity of the saints. 10. The beatitudes are so called because they show us how to be truly (\textbf{\textit{happy}}) in this life and in the next. 11. St. Paul mentions (\textbf{\textit{twelve}}) fruits of the Holy Ghost; and he contrasts them to the fruits of the (\textbf{\textit{flesh}}). 12. The beatitudes are excellent (\textbf{\textit{actions}}) (habits). 13. The least excellent of the Gifts is that called (\textbf{\textit{fear}}); the most excellent, that called (\textbf{\textit{wisdom}}). 14. Prudence is complemented by the Gift of (\textbf{\textit{counsel}}). 15. It is the Gift of (\textbf{\textit{knowledge}}) which makes us judge rightly of creatures. 16. Hence this Gift helps the virtue of (\textbf{\textit{hope}}). 17. It is the Gift of (\textbf{\textit{understanding}}) which gives us supernatural insight into God's revealed truths. 18. Therefore it assists the virtue of (\textbf{\textit{faith}}). 19. Which are the Gifts which give infused contemplation? These are (\textbf{\textit{wisdom}}), (\textbf{\textit{understanding}}) and (\textbf{\textit{knowledge}}). 20. Which give the highest form of contemplation? (\textbf{\textit{wisdom}}). 21. Which of the virtues does the Gift of Piety complement? (\textbf{\textit{religion}}) 22. To be truly happy, we must separate ourselves from what is sinful in (\textbf{\textit{creatures/ourselves}}) and unite ourselves to (\textbf{\textit{God}}). 23. We can be loyal to Christ only at the expense of (\textbf{\textit{suffering}}). 24. St. Paul speaks of us as ``Having nothing, and possessing (\textbf{\textit{all things}}).'' 25. When we have (\textbf{\textit{God}}), we have all.
\end{document}
